\begin{figure}[htb]
\hrule\smallskip
\begin{minipage}{0.7\hsize}
\begin{verbatim}
\c 1
\s Mawu ƒe nya to Mawu Vi la dzi va
\p
\v 1 Le blema la, Mawu ƒo nu na mía tɔgbuiwo zi
 geɖe to eƒe gbeƒãɖelawo dzi le mɔ vovovo nu.
\v 2 Ke le egbe ŋkeke mamlɛ siawo me la, Mawu ƒo nu
 na mí to Via dzi. Vi siae eya ŋutɔ tia be wòanyi
 nuwo katã dome. Eyama ke dzie Mawu to wɔ xexeame hã.
\v 3 Vi lae ɖe alesi Mawu ƒe ŋutikɔkɔe le la fia,
 eye Mawu ƒe nɔnɔme tututue le eya hã si.
 Vi lae tsɔ eƒe nya ƒe ŋusẽ la lé xexeme blibo
 la ɖe te. Esi eyama klɔ míaƒe nuvɔ̃wo ɖa vɔ la,
 eyi ɖabɔbɔ nɔ anyi ɖe Mawu Bubutɔgã si le dziƒo
 ʋĩ la ƒe nuɖusime.
\end{verbatim}
\end{minipage}\hfil
\begin{minipage}{0.3\hsize}
\font\sfont="Gentium Italic" at 8pt
\font\txfont="Gentium" at 8pt \txfont
\font\cfont="Gentium" at 22pt
\font\vfont="Gentium" at 5pt
\baselineskip=10pt
\centerline{\sfont Mawu ƒe nya to Mawu Vi la dzi va}
\smallskip
\def\v #1 {\leavevmode\raise2pt\hbox{\vfont #1}\kern1pt}
\setbox0=\hbox{\cfont 1\kern2pt}
\noindent\hangindent\wd0 \hangafter -2
\setbox0=\hbox{\lower\baselineskip\llap{\box0}}\dp0=0pt \box0
\v 1 Le blema la, Mawu ƒo nu na mía tɔgbuiwo zi
 geɖe to eƒe gbeƒãɖelawo dzi le mɔ vovovo nu.
\v 2 Ke le egbe ŋkeke mamlɛ siawo me la, Mawu ƒo nu
 na mí to Via dzi. Vi siae eya ŋutɔ tia be wòanyi
 nuwo katã dome. Eyama ke dzie Mawu to wɔ xexeame hã.
\v 3 Vi lae ɖe alesi Mawu ƒe ŋutikɔkɔe le la fia,
 eye Mawu ƒe nɔnɔme tututue le eya hã si.
 Vi lae tsɔ eƒe nya ƒe ŋusẽ la lé xexeme blibo
 la ɖe te. Esi eyama klɔ míaƒe nuvɔ̃wo ɖa vɔ la,
 eyi ɖabɔbɔ nɔ anyi ɖe Mawu Bubutɔgã si le dziƒo
 ʋĩ la ƒe nuɖusime.
\end{minipage}

\vskip10pt

\begin{minipage}{0.7\hsize}
%\mt پيدائش
\begin{verbatim}
\c 1
\s ‭دنيا ‭جي ‭پيدائش
\p
\v ‭1 ‭شروعات ‭۾ ‭خدا ‭زمين ‭۽ ‭آسمان ‭کي ‭پيدا ‭ڪيو. ‭
\v ‭2 ‭ان ‭وقت ‭زمين ‭بي​ترتيب ‭۽ ‭ويران ‭هئي. ‭اونهي ‭سمنڊ
جو ‭مٿاڇرو ‭اوندهہ ‭سان ‭ڍڪيل ‭هو ‭۽ ‭پاڻئَ ‭جي ‭مٿان ‭خدا
جي ‭روح ‭ڦيرا ‭پئي ‭ڪي
\v ‭3 ‭تڏهن ‭خدا ‭حڪم ‭ڏنو ‭تہ ‭”روشني ‭ٿئي.“ ‭سو ‭روشني ‭ٿي ‭پيئي. ‭
\end{verbatim}
\end{minipage}\hfil
\begin{minipage}{0.3\hsize}
\font\mtfont="Geeza Pro Bold" at 18pt
\font\sfont="Geeza Pro Bold" at 9pt
\font\txfont="Scheherazade:script=arab" at 11pt \txfont
\font\cfont="Geeza Pro Bold" at 24pt
\font\vfont="Scheherazade" at 7pt
\baselineskip=12pt \lineskiplimit=-10pt
%\centerline{\mtfont پيدائش}\vskip6pt
\centerline{\sfont دنيا جي پيدائش}
\smallskip
\def\v #1 {\leavevmode\raise3pt\hbox{\vfont #1}\kern1pt}
\setbox0=\hbox{\cfont ١\kern2pt}
\noindent\beginR\hangindent-\wd0 \hangafter -2
\setbox0=\hbox{\kern-\wd0\lower\baselineskip\box0}\dp0=0pt \box0
\def\x #1 {}
\v ١ شروعات ۾ خدا زمين ۽ آسمان کي پيدا ڪيو. 
\v ٢ ان وقت زمين بي​ترتيب ۽ ويران هئي. اونهي سمنڊ
جو مٿاڇرو اوندهہ سان ڍڪيل هو ۽ پاڻئَ جي مٿان خدا
جي روح ڦيرا پئي ڪي
\v ٣ تڏهن خدا حڪم ڏنو تہ ”روشني ٿئي.“ سو روشني ٿي پيئي. 
\end{minipage}
\smallskip\hrule

\caption{Using a \TeX\ macro package designed for Scripture formatting, showing examples in both African (extended Latin script) and Pakistani (Arabic script) languages.}
\label{fig-scripture}
\end{figure}
