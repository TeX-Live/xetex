% Copyright 2011 Karl Berry.
% You may freely use, modify, and/or distribute this file, without limitation.
% Experimentally determine what XeTeX finds with different syntaxes.

% fc-list : family style file spacing
% reports files et al.

\XeTeXtracingfonts = \maxdimen

% with quotes and no brackets, first looked up as otf/ttf, then as tfm.

% fc-list reports /usr/share/X11/fonts,
% XeTeX finds /usr/share/fonts/default/ghostscript/putr.pfa.  Why?
\font\1 = "Utopia" \1 utopia

% fc-list reports /usr/share/X11/fonts/Type1/cursor.pfa,
% XeTeX finds it there.  But it doesn't work, xdvipdfmx reports:
% ** ERROR ** Invalid font: -1 (0). Why?
\font\1 = "Cursor" \1 cursor

% this fails, presumably because it is only available as pcf:
%\font\1 = "Console" \1 console

% tfm ok:
\font\1 = "cmr10" \1 hello.


% with quotes and square brackets, looked up as otf/ttf/pfb.  This works:
\font\1 =  "[lmroman10-regular.otf]" at 16pt \1  lmotf

% This works, finds cmr10.pfb in tree (for output):
\font\1 =  "[cmr10]" at 16pt \1  lmtfm


% without quotes, always looked up as tfm.  Thus, this fails:
%\font\1 = lmroman10-regular.otf at 16pt \1  lm
\end

% From XeTeXFontMgr.h:
	PlatformFontRef					findFont(const char* name, char* variant, double ptSize);
		// 1st arg is name as specified by user (C string, UTF-8)
		// 2nd is /B/I/AAT/ICU[/USP]/S=## qualifiers
		// 1. try name given as "full name"
		// 2. if there's a hyphen, split and try "family-style"
		// 3. try as PostScript name
		// 4. try name as family with "Regular/Plain/Normal" style
		// apply style qualifiers and optical sizing if present
