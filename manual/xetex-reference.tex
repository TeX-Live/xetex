% Copyright 2007-2011 Will Robertson
% Copyright 2011 Karl Berry
% Copyright 2013 Khaled Hosny
%
% This work may be distributed and/or modified under the
% conditions of the LaTeX Project Public License, either version 1.3c
% of this license or (at your option) any later version.
% The latest version of this license is in
%   http://www.latex-project.org/lppl.txt
% and version 1.3 or later is part of all distributions of LaTeX
% version 2005/12/01 or later.
%
% This work has the LPPL maintenance status `maintained'.
%
% The Current Maintainer of this work is Khaled Hosny.
%
% This work consists of the files xetex-reference.tex,
% README.txt, and the derived file xetex-reference.pdf.

\documentclass[12pt]{article}

\suppressfontnotfounderror=1

\makeatletter
\def\@dotsep{999}

\usepackage{fontspec,unicode-math}
\setmainfont[Scale=MatchLowercase]{TeX Gyre Pagella}
\setsansfont[Scale=MatchLowercase]{TeX Gyre Heros}
\setmonofont[Scale=MatchLowercase, AutoFakeSlant=.2]{Inconsolata}
\setmathfont{TeX Gyre Pagella Math}

\usepackage{calc,fancyvrb,hyperref,refstyle,varioref,xcolor,hologo,xspace}
\usepackage{geometry}
%\geometry{screen,margin=3cm}

\newcommand\tex    {\hologo{TeX}\xspace}
\newcommand\xetex  {\hologo{XeTeX}\xspace}
\newcommand\pdftex {\hologo{pdfTeX}\xspace}
\newcommand\luatex {\hologo{LuaTeX}\xspace}
\newcommand\latex  {\hologo{LaTeX}\xspace}
\newcommand\context{\hologo{ConTeXt}\xspace}

\hypersetup{
  colorlinks,
  linkcolor=black,
  urlcolor=black,
  pdfsubject={The XeTeX reference guide},
  pdfauthor={Will Robertson \& Khaled Hosny},
  pdfkeywords={xetex, tex, typesetting, unicode, math, opentype, graphite, aat}
}

\usepackage[it]{titlesec}
\usepackage{enumitem}
\setlist{nolistsep}

\newenvironment{optdesc}
  {\begin{description}[font=\ttfamily,style=nextline,leftmargin=1.5cm]}
  {\end{description}}

\newcommand\cmd{%
  \noindent
  \begin{trivlist}\item[]
  \SaveVerb[%
    aftersave={%
      \begin{minipage}{\linewidth}
        \parindent=2em\relax\noindent
        \UseVerb{CMD}}]{CMD}}
\edef\|{|}
\DefineShortVerb{\|}
\newcommand\xarg[1]{$\langle\hbox{\rmfamily\itshape #1}\rangle$}
\def\<#1>{\xarg{#1}}
\newcommand\oarg[1]{\texttt{[\,#1\,]}}
\newcommand\desc[1]{\par\noindent\ignorespaces#1\par}
\def\endcmd{%
  \end{minipage}  
  \end{trivlist}}

\def\cs#1{\texttt{\textbackslash#1}}

\newsavebox\verbatimbox
\edef\examplefilename{\jobname.example}
\newlength\exampleindent
\setlength\exampleindent{1em}

\newenvironment{example}
  {\VerbatimEnvironment
   \begin{VerbatimOut}{\examplefilename}}
  {\end{VerbatimOut}
   \typesetexample}
   
\newcommand\typesetexample{%
  \begin{trivlist}\item[]
  \vrule
  \hspace{\exampleindent}
  \begin{minipage}{\linewidth-\exampleindent-\exampleindent}
    \textit{Example:}\par
    \vspace{0.4\baselineskip}
    \BVerbatimInput[fontsize=\small]{\examplefilename}\par
    \vspace{0.4\baselineskip}
    \color[rgb]{0.7,0,0}\input\examplefilename\relax
  \end{minipage}\par
  \end{trivlist}}

\let\strong\textbf
\newcommand\hlink[2]{\href{#1}{#2}\footnote{\url{#1}}}

\let\latin\textit
\def\eg{\latin{e.g.}}
\def\ie{\latin{i.e.}}
\def\Eg{\latin{E.g.}}
\def\Ie{\latin{I.e.}}
\def\etc{\@ifnextchar.{\latin{etc}}{\latin{etc.}\@}}

\def\opteq{\unskip\,\textcolor{gray}{[\textcolor{black}{=}]}\,}

\setlength\parskip{0pt}
\setlength\parindent{2em}
\raggedbottom

\begin{document}
\title{The \xetex reference guide}
\author{Will Robertson \and Khaled Hosny}
\maketitle

\vfill 

\section*{Introduction}

This document serves to summarise additional features of \xetex without
being so much as a ‘users’ guide. Note that much of the functionality
addressed here is provided in abstracted form in various \latex
packages and \context modules.

The descriptions here should be a fairly exhaustive list of the new
primitives and features of \xetex. Descriptions are still a little
aenemic, however.

\section*{License}

This work, is distributed under the terms of the LaTeX Project Public License
(\url{http://www.latex-project.org/lppl.txt}).

This basically means you are free to re-distribute this file as you
wish; you may also make changes to this file or use its contents for
another purpose, in which case you should make it clear, by way of a
name-change or some other means, that your changed version is a modified
version of the original. Please read the license text for more detailed
information.

\vfill\vfill\vfill\null

\newpage
\tableofcontents

\part{\xetex specifics}

\section{The \cs{font} command}

Traditionally, fonts were selected in \tex like this:
|\font\1=|\xarg{tfm name} with various options possibly appended
% harder to read without the extra space
such as \ ‘| at 10pt|’ \ or \ ‘| scaled 1.2|’, with obvious meaning. This syntax
still works, but it has been greatly extended in \xetex.

The extended syntax looks schematically like this:

{\centering
 |\font\1="|\xarg{font identifier}\xarg{font options}|:|\xarg{font features}|"|
 \xarg{\tex font options}\par}

\noindent The \xarg{font identifier} is the only mandatory part of the
above syntax.  If it is given in square brackets, (e.g.,
|[lmroman10-regular]|), it is taken as a font file name.
Without brackets, the name is looked up both as a file name and a system
font name.
When using a font name, the font is looked up through the operating
system, using (except on Mac~OS~X) the |fontconfig| library.  Running
|fc-list| should show you the font names available.  \Eg,
\begin{quote}\small
|\font\1="Liberation Serif"| \hfill \emph{look for OS-installed font}
\end{quote}

Fonts have many internal names, and XeTeX matches them in the following order:
\vskip1ex
\begin{itemize}
  \item Full Name;
  \item if the name has a hyphen, it is split into Family-Style pair then matched;
  \item PostScript Name;
  \item Family Name, if there is more than one match;
  \begin{itemize}
    \item look for font with “regular” bit set in OS/2 table, if no match;
    \item look for font with style “Regular”, “Plain”, “Normal” or “Roman”, in that order.
  \end{itemize}
\end{itemize}
\bigskip

When using a file name, the |xdvipdfmx| driver must be used (this is the
default). The current directory and the |texmf| trees are searched for
files matching the name, or the path may be embedded in the font
declaration, as usual with |kpathsea|. \Eg,
\begin{quote}\small
|\font\2="[lmroman10-regular]"| \hfill
  {\em find |lmroman10-regular.otf| in any tree}
|\font\3="[/myfonts/fp9r8a]"| \hfill
  {\em look for |fp9r8a| only in |/myfonts/|}
\end{quote}

A file with either an |.otf|, |.ttf| or |.pfb| extension (in that order) will be found.  The
extension can also be specified explicitly.
If the file is a font collection (e.g., |.ttc| or |.dfont|), the index of the
font can be specified using a colon followed by zero-based font index inside
the square brackets. \Eg,
\begin{quote}\small
|\font\4="[myfont.ttc:1]"| \hfill {\em load the second font from |myfont.ttc| file}
\end{quote}


\subsection{Font options}

\xarg{Font options} are only applicable when the font is selected
through the operating system (\ie, without square brackets).  They may
be any concatenation of the following:

\begin{optdesc}
\item[/B] Use the bold version of the selected font.
\item[/I] Use the italic version of the selected font.
\item[/BI] Use the bold italic version of the selected font.
\item[/IB] Same as \texttt{/BI}.
\item[/S=$x$] Use the version of the selected font corresponding to the
optical size $x$\,pt.
\item[/AAT] Explicitly use the AAT renderer (Mac~OS~X only).
\item[/OT] Explicitly use the OpenType renderer (new in 0.9999).
\item[/GR] Explicitly use the Graphite renderer.%
           \footnote{\url{http://scripts.sil.org/cms/scripts/page.php?site_id=projects&item_id=graphite_home}}
\item[/ICU] Explicitly use the OpenType renderer (deprecated since 0.9999).
\end{optdesc}


\subsection{Font features}

The \xarg{font features} is a comma or semi-colon separated list
activating or deactivating various OpenType, Graphite, or AAT font
features, which will vary by font.  In contrast to font options,
features work whether the font is selected by file name or through the
operating system.

The \xetex documentation files \path{aat-info.tex} and
\path{opentype-info.tex} provide per-font lists of supported features.

\subsubsection{Arbitrary OpenType, Graphite, or AAT features}

OpenType font features are chosen with
\hlink{http://www.microsoft.com/typography/otspec/featuretags.htm}{standard
tags}. They may be either comma- or semicolon-separated, and prefixed
with a |+| to turn them on and a |-| to turn them off, optionally followed
by |=| and a 0-based index for selecting alternates from multiple
alternates features (ignored for |-| prefixed tags).

\begin{example}
\font\liber="Linux Libertine O/I=5:+smcp" at 12pt
\liber This is the OpenType font Linux Libertine in italic with small caps.
\end{example}

Varying depending on the language and script in use (see
\secref[vref]{script}), a small number of OpenType features, if they
exist, will be activated by default.

\begin{example}
\font\antt="Antykwa Torunska"         at 12pt \antt 0
\font\antt="Antykwa Torunska:+aalt=0" at 12pt \antt 0
\font\antt="Antykwa Torunska:+aalt=1" at 12pt \antt 0
\font\antt="Antykwa Torunska:+aalt=2" at 12pt \antt 0
\font\antt="Antykwa Torunska:+aalt=3" at 12pt \antt 0
\font\antt="Antykwa Torunska:+aalt=4" at 12pt \antt 0
\end{example}

AAT font features and Graphite font features are specified by strings
within each font rather than standardised tags. Therefore, even
equivalent features between different fonts can have different names.

\begin{example}
\font\gra="Charis SIL/GR:Small Caps=True" at 12pt
\gra This is the Graphite font Charis SIL with small caps.
\end{example}

\subsubsection{Options for all fonts}

Some font features may be applied for any font. These are
\begin{optdesc}
\item[mapping=\textsl{<font map>}] 
Uses the specified font mapping for this font. This uses the TECKit
engine to transform unicode characters in the last-minute processing
stage of the source. For example, |mapping=tex-text| will enable the
classical mappings from ugly ascii |``---''| to proper typographical
glyphs “—”, and so on.

\item[color={\slshape RRGGBB}{[{\slshape TT}]}] 
Triple pair of hex values to specify the colour in RGB space, with an
optional value for the transparency.

\item[letterspace=$x$] 
Adds $x/S$ space between letters in words, where $S$ is the font size.

\item[embolden=$x$]
Increase the envelope of each glyph by the set amount (this makes the
letters look ‘more bold’). $x=0$ corresponds to no change; $x=1.5$ is a
good default value.

\item[extend=$x$]
Stretch each glyph horizontally by a factor of $x$ (i.e., $x=1$
corresponds to no change).

\item[slant=$x$]
Slant each glyph by the set amount. $x=0$ corresponds to no change;
$x=0.2$ is a good default value. The slant is given by $x=R/S$ where $R$
is the displacement of the top edge of each glyph and $S$ is the point
size.

\end{optdesc}

\subsubsection{OpenType script and language support}\seclabel{script}

OpenType font features (and font behaviour) can vary by
\hlink{http://www.microsoft.com/typography/otspec/scripttags.htm}{script}
(‘alphabet’) and by
\hlink{http://www.microsoft.com/typography/otspec/languagetags.htm}{language}.
These are selected with four and three letter tags, respectively.

\begin{optdesc}
\item[script=\textsl{<script tag>}] Selects the font script.
\item[language=\textsl{<lang tag>}] Selects the font language.
\end{optdesc}

\subsubsection{Multiple Master and Variable Axes AAT font support}

\begin{optdesc}
\item[weight=$x$] Selects the normalised font weight, $x$.
\item[width=$x$] Selects the normalised font width, $x$.
\item[optical size=$x$] Selects the optical size, $x$\,pt. Note the
difference between the \texttt{/S} font option, which selects discrete
fonts.
\end{optdesc}

\subsubsection{Vertical typesetting}
\begin{optdesc}
\item[vertical] 
Enables glyph rotation in the output so vertical typesetting can be performed.
\end{optdesc}

\part{New commands}

\section{Font primitives}

\cmd|\XeTeXtracingfonts|
\desc{If nonzero, reports where fonts are found in the log file.}
\endcmd

\cmd|\XeTeXfonttype|
\xarg{font}
\desc{
  Expands to a number corresponding to which renderer is used for a
  \xarg{font}:
  \begin{optdesc}
    \item [0] for \tex (a legacy TFM-based font);
    \item [1] for AAT;
    \item [2] for OpenType;
    \item [3] for Graphite.
  \end{optdesc}
}
\endcmd

\begin{example}
\newcommand\whattype[1]{%
  \texttt{\fontname#1} is rendered by
  \ifcase\XeTeXfonttype#1\TeX\or AAT\or OpenType\or Graphite\fi.\par}
\font\1="cmr10"
\font\2="Charis SIL"
\font\3="Charis SIL/OT"
\whattype\1 \whattype\2 \whattype\3
\end{example}

\cmd|\XeTeXfirstfontchar|
\xarg{font}
\desc{Expands to the code of the first character in \xarg{font}.}
\endcmd

\cmd|\XeTeXlastfontchar|
\xarg{font}
\desc{Expands to the code of the last character in \xarg{font}.}
\endcmd

\begin{example}
\font\1="Charis SIL"\1
The first character in Charis SIL is: "\char\XeTeXfirstfontchar\1"
and the last character is: "\char\XeTeXlastfontchar\1".
\end{example}

\cmd|\XeTeXglyph|
\xarg{glyph slot}
\desc{Inserts the glyph in \xarg{glyph slot} of the current font. \strong{Font
specific}, so will give different output for different fonts and
possibly even different versions of the same font.}
\endcmd

\cmd|\XeTeXcountglyphs|
\xarg{font}
\desc{The count of the number of glyphs in the specified \xarg{font}.}
\endcmd

\cmd|\XeTeXglyphname|
\xarg{font}
\xarg{glyph slot}
\desc{Expands to the name of the glyph in \xarg{glyph slot} of \xarg{font}.
\strong{Font specific}, so will give different output for different fonts and
possibly even different versions of the same font.}
\endcmd

\cmd|\XeTeXglyphindex|
|"|\xarg{glyph name}|"| \xarg{space} \emph{or} \cs{relax}
\desc{Expands to the glyph slot corresponding to the (possibly
font specific) \xarg{glyph name} in the currently selected font. Only
works for TrueType fonts (or TrueType-based OpenType fonts) at
present. Use \texttt{fontforge} or similar to discover glyph names.}
\endcmd

\cmd|\XeTeXcharglyph|
\xarg{char code} 
\desc{Expands to the default glyph number of character \xarg{char code}
in the current font, or 0 if the character is not available in the
font.}
\endcmd

\begin{example}
\font\1="Charis SIL"\1
The glyph slot in Charis SIL for the Yen symbol is:
    \the\XeTeXglyphindex"yen" . % the font-specific glyph name
Or: \the\XeTeXcharglyph"00A5.   % the unicode character slot

This glyph may be typeset with the font-specific glyph slot:
\XeTeXglyph150, 
or the unicode character slot:
\char"00A5.
\end{example}

\cmd|\XeTeXglyphbounds|
\xarg{edge} \xarg{glyph slot}
\desc{Expands to a dimension corresponding to one of the bounds of a
glyph, where \xarg{edge} is an integer from 1 to~4 indicating the
left/top/right/bottom edge respectively, and \xarg{glyph slot} is an
integer glyph index in the current font (only valid for non TFM-based
fonts).

The left and right measurements are the glyph sidebearings, measured
‘inwards’ from the origin and advance respectively, so for a glyph that
fits completely within its ‘cell’ they will both be positive; for a
glyph that ‘overhangs’ to the left or right, they will be negative. The
actual width of the glyph’s bounding box, therefore, is the character
width (advance) minus both these sidebearings.

The top and bottom measurements are measured from the baseline, like
\tex’s height and depth; the height of the bounding box is the sum of
these two dimensions.}
\endcmd

\begin{example}
\def\shadebbox#1{%
\leavevmode\rlap{%
  \dimen0=\fontcharwd\font`#1%
  \edef\gid{\the\XeTeXcharglyph`#1}%
  \advance\dimen0 by -\XeTeXglyphbounds1 \gid
  \advance\dimen0 by -\XeTeXglyphbounds3 \gid
  \kern\XeTeXglyphbounds1 \gid
  \special{color push rgb 1 1 0.66667}%
  \vrule width \dimen0
         height \XeTeXglyphbounds2 \gid
         depth \XeTeXglyphbounds4 \gid
  \special{color pop}%
  \kern\XeTeXglyphbounds3 \gid}%
  #1}
	
\noindent
\font\x="Charis SIL/I" at 24pt \x
\shadebbox{A} \shadebbox{W} \shadebbox{a} \shadebbox{f}
\shadebbox{;} \shadebbox{*} \shadebbox{=}
\end{example}

\cmd|\XeTeXuseglyphmetrics|
\desc{Counter to specify if the height and depth of characters are taken
into account while typesetting ($\ge\mathtt1$). Otherwise ($<\mathtt1$),
a single height and depth for the entire alphabet is used. Gives better
output but is slower. Activated ($\ge\mathtt1$) by default.}
\endcmd

\begin{example}
\XeTeXuseglyphmetrics=0 \fbox{a}\fbox{A}\fbox{j}\fbox{J} vs.
\XeTeXuseglyphmetrics=1 \fbox{a}\fbox{A}\fbox{j}\fbox{J}
\end{example}

\subsection{OpenType fonts}

\cmd|\XeTeXOTcountscripts|
\xarg{font}
\desc{Expands to the number of scripts in the \xarg{font}.}
\endcmd

\cmd|\XeTeXOTscripttag|
\xarg{font}
\xarg{integer, $n$}
\desc{Expands to the $n$-th script tag of \xarg{font}.}
\endcmd

\cmd|\XeTeXOTcountlanguages|
\xarg{font}
\xarg{script tag}
\desc{Expands to the number of languages in the script of \xarg{font}.}
\endcmd

\cmd|\XeTeXOTlanguagetag|
\xarg{font}
\xarg{script tag}
\xarg{integer, $n$}
\desc{Expands to the $n$-th language tag in the script of \xarg{font}.}
\endcmd

\cmd|\XeTeXOTcountfeatures|
\xarg{font}
\xarg{script tag}
\xarg{language tag}
\desc{Expands to the number of features in the language of a script of \xarg{font}.}
\endcmd

\cmd|\XeTeXOTfeaturetag|
\xarg{font}
\xarg{script tag}
\xarg{language tag}
\xarg{integer, $n$}
\desc{Expands to the $n$-th feature tag in the language of a script of \xarg{font}.}
\endcmd


\subsection{AAT and Graphite fonts}

\subsubsection{Features}

\cmd|\XeTeXcountfeatures|
\xarg{font}
\desc{Expands to the number of features in the \xarg{font}.}
\endcmd

\cmd|\XeTeXfeaturecode|
\xarg{font}
\xarg{integer, $n$}
\desc{Expands to the feature code for the $n$-th feature in the \xarg{font}.}
\endcmd

\cmd|\XeTeXfeaturename|
\xarg{font}
\xarg{feature code}
\desc{Expands to the name corresponding to the \xarg{feature code} in
the \xarg{font}.}
\endcmd

\cmd|\XeTeXisexclusivefeature|
\xarg{font}
\xarg{feature code}
\desc{Expands to a number greater than zero if the feature of a font is
exclusive (can only take a single selector).}
\endcmd

\cmd|\XeTeXfindfeaturebyname|
\xarg{font}
\xarg{feature name}
\desc{This command provides a method to query whether a feature name
corresponds to a feature contained in the font. It represents an integer
corresponding to the feature number used to access the feature
numerically. If the feature does not exist, the integer is
\texttt{-1}. Also see \cs{XeTeXfindselectorbyname}.}
\endcmd

\begin{example}
\font\1="Charis SIL/GR" at 10pt
\def\featname{Uppercase Eng alternates}
The feature ‘\featname’ has index
\the\XeTeXfindfeaturebyname\1 "\featname"\relax
\end{example}

\subsubsection{Feature selectors}

\cmd|\XeTeXcountselectors|
\xarg{font}
\xarg{feature}
\desc{Expands to the number of selectors in a \xarg{feature} of a \xarg{font}.}
\endcmd

\cmd|\XeTeXselectorcode|
\xarg{font}
\xarg{feature code}
\xarg{integer, $n$}
\desc{Expands to the selector code for the $n$-th selector in a
\xarg{feature} of a \xarg{font}.}
\endcmd

\cmd|\XeTeXselectorname|
\xarg{font}
\xarg{feature code}
\xarg{selector code}

\desc{Expands to the name corresponding to the \xarg{selector code} of a
feature of a \xarg{font}.}
\endcmd

\cmd|\XeTeXisdefaultselector|
\xarg{font}
\xarg{feature code}
\xarg{selector code}
\desc{Expands to a number greater than zero if the selector of a feature
of a font is on by default.}
\endcmd


\cmd|\XeTeXfindselectorbyname|
\xarg{font}
\xarg{feature name}
\xarg{selector name}
\desc{This command provides a method to query whether a feature selector
name corresponds to a selector of a specific feature contained in the
font. It represents an integer corresponding to the selector number used
to access the feature selector numerically. If the feature selector does
not exist, the integer is \texttt{-1}.

The indices given by this command and by \cs{XeTeXfindfeaturebyname} can
be used in Graphite fonts to select font features directly (see example
below). Alternatively, they can be used as a means of checking whether a
feature/selector exists before attempting to use it.}
\endcmd

\begin{example}
\font\1="Charis SIL/GR" at 10pt
\def\featname{Uppercase Eng alternates}
\newcount\featcount
\featcount=\XeTeXfindfeaturebyname\1 "\featname"\relax

\def\selecname{Large eng on baseline}
\newcount\seleccount
\seleccount=\XeTeXfindselectorbyname\1 \featcount "\selecname"\relax
The feature selector ‘\selecname’ has index \the\seleccount

\font\2="Charis SIL/GR:\featname=\selecname" at 10pt
\font\3="Charis SIL/GR:\the\featcount=\the\seleccount" at 10pt

Activating the feature: \1 Ŋ \2 Ŋ \3 Ŋ
\end{example}

\subsubsection{Variation axes}

\cmd|\XeTeXcountvariations|
\xarg{font}
\desc{Expands to the number of variation axes in the \xarg{font}.}
\endcmd

\cmd|\XeTeXvariation|
\xarg{font}
\xarg{integer, $n$}
\desc{Expands to the variation code for the $n$-th feature in the \xarg{font}.}
\endcmd

\cmd|\XeTeXvariationname|
\xarg{font}
\xarg{variation code}
\desc{Expands to the name corresponding to the \xarg{feature code} in
the \xarg{font}.}
\endcmd

\cmd|\XeTeXvariationmin|
\xarg{font}
\xarg{variation code}
\desc{Expands to the minimum value of the variation corresponding to the
\xarg{variation code} in the \xarg{font}.}
\endcmd

\cmd|\XeTeXvariationmax|
\xarg{font}
\xarg{variation code}
\desc{Expands to the maximum value of the variation corresponding to the
\xarg{variation code} in the \xarg{font}.}
\endcmd

\cmd|\XeTeXvariationdefault|
\xarg{font}
\xarg{variation code}
\desc{Expands to the default value of the variation corresponding to the
\xarg{variation code} in the \xarg{font}.}
\endcmd

\cmd|\XeTeXfindvariationbyname|
\xarg{font}
\xarg{variation name}
\desc{An integer corresponding to the internal index corresponding to
the \xarg{variation name}. This index cannot be used directly but may be
used to error-check that a specified variation name exists before
attempting to use it.}
\endcmd

\subsection{Maths fonts}

The primitives described following are extensions of \tex’s 8-bit primitives.

In the following commands, \xarg{fam.} is a number (0–255) representing
font to use in maths. \xarg{math type} is the 0–7 number corresponding
to the type of math symbol; see a \tex reference for details.

Before version 0.9999.0 the following primitives had |\XeTeX| prefix instead of
|\U|, the old names are deprecated and will be removed in the future.

\cmd|\Umathcode|
\xarg{char slot}
\opteq
\xarg{math type}
\xarg{fam.}
\xarg{glyph slot}
\desc{Defines a maths glyph accessible via an input character. Note that
the input takes \emph{three} arguments unlike \tex’s \cs{mathcode}.}
\endcmd

\cmd|\Umathcodenum|
\xarg{char slot}
\opteq
\xarg{math type/fam./glyph slot}
\desc{Pure extension of \cs{mathcode} that uses a ‘bit-packed’ single
number argument. Can also be used to extract the bit-packed mathcode
number of the \xarg{char slot} if no assignment is given.}
\endcmd

\cmd|\Umathchar|
\xarg{math type}
\xarg{fam.}
\xarg{glyph slot}
\desc{Typesets the math character in the \xarg{glyph slot} in the family
specified.}
\endcmd

\cmd|\Umathcharnum|
\xarg{type/fam./glyph slot}
\desc{Pure extension of \cs{mathchar} that uses a ‘bit-packed’ single
number argument. Can also be used to extract the bit-packed mathcode
number of the \xarg{char slot} if no assignment is given.}
\endcmd

\cmd|\Umathchardef|
\xarg{control sequence}
\opteq
\xarg{math type}
\xarg{fam.}
\xarg{glyph slot}
\desc{Defines a maths glyph accessible via a control sequence.}
\endcmd

\cmd|\Umathcharnumdef|
\xarg{control sequence}
\opteq
\xarg{type/fam./glyph slot}
\desc{Defines a control sequence for accessing a maths glyph using the
‘bit-packed’ number output by, e.g., \cs{Umathcodenum}. This would
be used to replace legacy code such as
\cs{mathchardef}\cs{foo}\texttt{=}\cs{mathcode}\texttt{`}\cs{\-}.}
\endcmd

\cmd|\Udelcode|
\xarg{char slot}
\opteq
\xarg{fam.}
\xarg{glyph slot}
\desc{Defines a delimiter glyph accessible via an input character.}
\endcmd

\cmd|\Udelcodenum|
\xarg{char slot}
\opteq
\xarg{fam./glyph slot}
\desc{Pure extension of \cs{delcode} that uses a ‘bit-packed’ single
number argument. Can also be used to extract the bit-packed delcode
number of the \xarg{char slot} if no assignment is given.}
\endcmd

\cmd|\Udelimiter|
\xarg{math type}
\xarg{fam.}
\xarg{glyph slot}
\desc{Typesets the delimiter in the \xarg{glyph slot} in the family
specified of either \xarg{math type} 4 (opening) or 5 (closing).}
\endcmd

\cmd|\Umathaccent|
\oarg{keyword}
\xarg{math type}
\xarg{fam.}
\xarg{glyph slot}
\desc{Typesets the math accent character in the \xarg{glyph slot} in the
family specified. Starting from version 0.9998, \cs{Umathaccent} accepts
optional keyword:\medskip
\begin{optdesc}
\item[fixed] Don’t stretch the accent, the default is to stretch it:
             $\widehat{M}$ vs $\hat{M}$.
\item[bottom] Place the accent below its base. Can be followed by the
             \texttt{fixed} keyword.
\end{optdesc}}
\endcmd

\cmd|\Uradical|
\xarg{fam.}
\xarg{glyph slot}
\desc{Typesets the radical in the \xarg{glyph slot} in the family specified.}
\endcmd

\section{Character classes}

The idea behind character classes is to define a boundary where tokens
can be added to the input stream without explicit markup. It was
originally intended to add glue around punctuation to effect correct
Japanese typesetting. This feature can also be used to adjust space
around punctuation for European traditions. The general nature of this
feature, however, lends it to several other useful applications
including automatic font switching when small amounts of another
language (in another script) is present in the text.

\cmd|\XeTeXinterchartokenstate|
\desc{Counter. If positive, enables the character classes functionality.}
\endcmd

\cmd|\newXeTeXintercharclass|
\xarg{control sequence}
\desc{Allocates a new interchar class and assigns it to the
\xarg{control sequence} argument.}
\endcmd

\cmd|\XeTeXcharclass|
\xarg{char slot}
\opteq
\xarg{interchar class}
\desc{Assigns a class corresponding to \xarg{interchar class} (range
0–255) to a \xarg{char slot}. Most characters are class 0 by
default. Class 1 is for CJK ideographs, classes 2 and 3 are CJK
punctuation. The boundary of a text string is considered class 255,
wherever there is a boundary between a ‘run’ of characters and something
else — glue, kern, math, box, etc. Special case class 256 is ignored;
useful for diacritics so I’m told.}
\endcmd

\cmd|\XeTeXinterchartoks|
\xarg{interchar class 1}
\xarg{interchar class 2}
\opteq
|{|\xarg{token list}|}|
\desc{Defines tokens to be inserted between
\xarg{interchar class 1} and \xarg{interchar class 2} (in that order).}
\endcmd

\begin{example}
\XeTeXinterchartokenstate = 1
\newXeTeXintercharclass \mycharclassa
\newXeTeXintercharclass \mycharclassA
\newXeTeXintercharclass \mycharclassB
\XeTeXcharclass `\a \mycharclassa
\XeTeXcharclass `\A \mycharclassA
\XeTeXcharclass `\B \mycharclassB

% between "a" and "A":
\XeTeXinterchartoks \mycharclassa \mycharclassA = {[\itshape}
\XeTeXinterchartoks \mycharclassA \mycharclassa = {\upshape]}

% between " " and "B":
\XeTeXinterchartoks 255 \mycharclassB = {\bgroup\color{blue}}
\XeTeXinterchartoks \mycharclassB 255 = {\egroup}

% between "B" and "B":
\XeTeXinterchartoks \mycharclassB \mycharclassB = {.}

aAa A a B aBa BB
\end{example}

\noindent In the above example the input text is typeset as\par
{\centering
 \verb|a[\itshape A\upshape]a A a \bgroup\color{blue}B\egroup aBa B.B|\par}

\newpage
\section{Encodings}

\cmd|\XeTeXinputnormalization|
\xarg{Integer}
\desc{Specify whether \xetex is to perform normalisation on the input
text and, if so, what type of normalisation to use. See
\url{http://unicode.org/reports/tr15/} for a description of Unicode
normalisation. The \<Integer> value can be:\medskip
\begin{optdesc}
\item[0] (default) do not perform normalisation.
\item[1] normalise to NFC form, using precomposed characters where possible
         instead base characters with combining marks.
\item[2] normalise to NFD form, using base characters with combining marks
         instead of precomposed characters.
\end{optdesc}}
\endcmd

\cmd|\XeTeXinputencoding|
\xarg{Charset name}
\desc{Defines the input encoding of the following text.}
\endcmd

\cmd|\XeTeXdefaultencoding|
\xarg{Charset name}
\desc{Defines the input encoding of subsequent files to be read.}
\endcmd

\section{Line breaking}

\cmd|\XeTeXdashbreakstate|
\xarg{Integer}
\desc{Specify whether line breaks after en- and em-dashes are
allowed. Off, 0, by default.}
\endcmd

\cmd|\XeTeXlinebreaklocale|
\xarg{Locale ID}
\desc{Defines how to break lines for multilingual text.}
\endcmd

\cmd|\XeTeXlinebreakskip|
\xarg{Glue}
\desc{Inter-character linebreak stretch}
\endcmd

\cmd|\XeTeXlinebreakpenalty|
\xarg{Integer}
\desc{Inter-character linebreak penalty}
\endcmd

\cmd|\XeTeXupwardsmode|
\xarg{Integer} 
\desc{If greater than zero, successive lines of text (and rules, boxes,
etc.) will be stacked upwards instead of downwards.}
\endcmd

\section{Graphics}

Thanks to Heiko Oberdiek, Paul Isambert, and William Adams for their
help with the documentation in this section.

\cmd|\XeTeXpicfile|
\xarg{filename}
\oarg{ scaled \xarg{int} \|
  xscaled \xarg{int} \|
  yscaled \xarg{int} \| \\\hspace*{2em}
  width \xarg{dimen} \|
  height \xarg{dimen} \|
  rotated \xarg{decimal} }
\desc{Insert an image. See below for explanation of optional arguments.}
\endcmd

\cmd|\XeTeXpdffile|
\xarg{filename}
\oarg{page \xarg{int}}
\oarg{ crop \| media \| bleed \| trim \| art }\\\hspace*{2em}
\oarg{scaled \xarg{int} \|
  xscaled \xarg{int} \|
  yscaled \xarg{int} \|
  width \xarg{dimen} \|\\\hspace*{2em}~
  height \xarg{dimen} \|
  rotated \xarg{decimal}}
\desc{Insert (pages of) a PDF. See below for explanation of optional
arguments.}
\endcmd

\noindent In the graphic/PDF commands above, \xarg{filename} is the
usual file name argument of \cs{input}, \cs{openin}, \etc.  It must not
terminated by \cs{relax} if options are given.  \xarg{int} and
\xarg{dimen} are the usual integer or dimen specifications of regular
\tex.

The rotation is specified in degrees (\ie, an input of ‘|360|’ is full
circle) and the rotation is counterclockwise. The syntax of
\xarg{decimal} require some explanation:
\begin{quote}
\<decimal> $\to$ \<optional signs>\<unsigned decimal>\\
\<unsigned decimal> $\to$ \<normal decimal>
  \| \<coerced dimen> \| \<internal dimen>\\
\<normal decimal> $\to$ \<normal integer> \| \<decimal constant>
\end{quote}
A \xarg{coerced dimen} or \xarg{internal dimen} is interpreted as number
with unit ‘|pt|’. For example, for a rotation specified with a dimension
\cs{testdim},
\begin{itemize}
\item \verb|\testdim=45pt  | results in a rotation of 45\textdegree,
\item \verb|\testdim=1in    | is 72.27\textdegree, and
\item \verb|\testdim=100sp| is (100/65536)\textdegree.
\end{itemize}
In all cases the resulting decimal number for rotation $x$ must be  
within the limits $-16384 < x < 16384$.

The \cs{XeTeXpdffile} command takes one more optional argument for
specifying to which ‘box’ the PDF should be cropped before inserting
it (the second optional argument listed in thes syntax of
\cs{XeTeXpdffile} above). The PDF standard defines a number of
(rectangular) bounding boxes that may be specified for various
purposes. These are described in the PDF Standard\footnote{Adobe
Systems Incorporated, 2008:\\
\url{http://www.adobe.com/devnet/acrobat/pdfs/PDF32000_2008.pdf}} and
summarised below.
\begin{quote}
\begin{description}[style=nextline,leftmargin=1.5cm]
\item [media] the box defining the physical page size.
\item [crop] the box of the page contents for display/printing purposes.
\item [bleed] the box containing the page contents plus whatever extra
space required for printing purposes.
\item [trim] the box of the finished page after trimming the printed
‘bleed box’.
\item [art] the box containing the ‘meaningful content’ of the
page. This could be the crop box with boilerplate text/logos trimmed
off.
\end{description}
\end{quote}
When not specified in the PDF to be inserted, the crop box defaults
to the media box, and the bleed, trim, and art boxes default to the crop
box.

\cmd|\XeTeXpdfpagecount|
\xarg{filename}
\desc{Expands to the number of pages in a PDF file.}
\endcmd


\section{Character protrusion}

\cmd|\XeTeXprotrudechars|
\xarg{integer}
\desc{Equivalent to \cs{pdfprotrudechars} in pdf\TeX{} for controlling
character protrusion or ‘margin kerning’. When set to zero (default), character
protrusion is turned off. When set to one, it is activated but will not affect
line-breaking. When set to two, line-breaking decisions will change as a result
of the character protrusion.}
\begin{example}
\XeTeXprotrudechars=2
\font\rm="[texgyrepagella-regular.otf]"\relax
\rm
\rpcode\font\XeTeXcharglyph\hyphenchar\font=250
\hsize=20mm
a a a a a a a a a abbabbabb aabbabbabb abbabb
\end{example}
See the pdf\TeX{} documentation for further details.
\endcmd

\cmd|\rpcode| \xarg{font} \xarg{char slot} (integer, $n$)
\desc{Sets the right-side character protrusion value of the \xarg{char slot} in
the specified \xarg{font} to $n/1000$\,em. $n$ is clipped to $\pm1000$.}
\endcmd

\cmd|\lpcode| \xarg{font} \xarg{char slot} (integer, $n$)
\desc{Sets the left-side character protrusion value of the \xarg{char slot} in
the specified \xarg{font} to $n/1000$\,em. $n$ is clipped to $\pm1000$.}
\endcmd


\section{Cross-compatibility with \pdftex and/or \luatex}

\cmd|\pdfpageheight|
\xarg{dimension}
\desc{The height of the PDF page.}
\endcmd

\cmd|\pdfpagewidth|
\xarg{dimension}
\desc{The width of the PDF page.}
\endcmd

\cmd|\pdfsavepos|
\desc{Saves the current location of the page in the typesetting stream.}
\endcmd

\cmd|\pdflastxpos|
\desc{Retrieves the horizontal position saved by \texttt{\char`\\pdfsavepos}.}
\endcmd

\cmd|\pdflastypos|
\desc{Retrieves the vertical position saved by \texttt{\char`\\pdfsavepos}.}
\endcmd

\cmd|\ifincsname...(\else...)\fi|
\desc{\tex conditional to branch true if the expansion occurs within
\texttt{\char`\\csname ... \char`\\endcsname}.}
\endcmd

\begin{example}
\def\x{\ifincsname y\else hello\fi}
\def\y{goodbye}
\x/\csname\x\endcsname
\end{example}

\cmd|\ifprimitive| \xarg{control sequence} |...(\else...)\fi|
\desc{\tex conditional to test if a control sequence is a primitive and
that it has not been redefined.}
\endcmd

\cmd|\primitive|
\xarg{control sequence}
\desc{If the control sequence is a primitive that’s been redefined, this
command causes it to expand with its original (i.e., primitive)
definition.}
\endcmd

\cmd|\shellescape|
\desc{Read-only status indicating the level of ‘shell escape’
allowed. That is, whether commands are allowed to be executed through
\texttt{\char`\\write18\char`\{...\char`\}}. Expands to zero for off;
one for on (allowed); two is ‘restricted’ (default in TeX Live 2009 and
greater) in which a subset of commands only are allowed.}
\endcmd

\begin{example}
Shell escape \ifnum\shellescape>0 is \else is not \fi enabled.
\end{example}

\cmd|\strcmp|
\xarg{arg one}
\xarg{arg two}
\desc{Compares the full expansion of the two token list
arguments. Expands to zero if they are the same, less than one if the
first argument sorts lower (lexicographically) than the second argument,
and greater than one if vice versa.}
\endcmd

\begin{example}
‘a’ is less than ‘z’: \strcmp{a}{z}

\def\z{a}
The tokens expand before being compared: \strcmp{a}{\z}

\def\a{z}
Therefore, |\a| is greater than |\z|: \strcmp{\a}{\z}

\edef\b{\string b}
Also note that catcodes are ignored: \strcmp{b}{\b}
\end{example}

\cmd|\suppressfontnotfounderror|
\xarg{integer}
\desc{When set to zero (default) if a font is loaded that cannot be
located by \xetex, an error message results and typesetting is
halted. When set to one, this error message is
suppressed and the font control sequence being defined is set to
\cs{nullfont}.}
\begin{example}
\suppressfontnotfounderror=1
\font\x="ImpossibleFont" at 10pt
\ifx\x\nullfont
  \font\x="Georgia" at 10pt
\fi
\x This would be ‘ImpossibleFont’, if it existed.
\end{example}
\endcmd


\section{Misc.}

\cmd|\XeTeXversion|
\desc{Expands to a number corresponding to the \xetex version.}
\endcmd

\cmd|\XeTeXrevision|
\desc{Expands to a string corresponding to the \xetex revision number.}
\endcmd

\begin{example}
The \xetex version used to typeset this document is:
\the\XeTeXversion\XeTeXrevision
\end{example}

\end{document}
