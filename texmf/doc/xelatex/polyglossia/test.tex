\documentclass[a4paper]{article}
\usepackage[no-math]{fontspec}
% makes fontspec very quiet!
\makeatletter
\renewcommand\zf@PackageWarning[1]{\relax}%
\makeatother
\usepackage{xltxtra,url}
\usepackage{polyglossia}
\setdefaultlanguage{french}
%\load{german} %[spelling=new]
\setotherlanguage[dialect=british]{english}
\setotherlanguage[variant=poly]{greek}
\setotherlanguages{german,latin,russian,turkish,polish,latvian}
\setromanfont{Linux Libertine}
\defaultfontfeatures{Scale=MatchLowercase}
\setmonofont{Inconsolata}
\setsansfont{Gill Sans Std}
%\newfontfamily\russianfont[Script=Cyrillic,Language=Russian]{}
\begin{document}
\hyphenation{Bru-xel-les}

\textbf{Le français}\footnote{ From \url{http://fr.wikipedia.org/wiki/Français}} est une langue romane parlée en France, dont elle est originaire (la «langue d'oïl»), ainsi qu'en Afrique francophone, au Canada (principalement au Québec, au Nouveau-Brunswick et en Ontario), en Belgique (en Région wallonne et à Bruxelles), en Suisse, au Liban, en Haïti et dans d'autres régions du monde, soit au total dans 51 pays du monde ayant pour la plupart fait partie des anciens empires coloniaux français et belge.
(Aujourd'hui: \today!)
\bigskip

\begin{english}
\textbf{English}\footnote{From \url{http://en.wikipedia.org/wiki/English_language}} is a West Germanic language originating in England, and the first language for most people in Australia, Canada, the Commonwealth Caribbean, Ireland, New Zealand, the United Kingdom and the United States of America (also commonly known as the Anglosphere). It is used extensively as a second language and as an official language throughout the world, especially in Commonwealth countries and in many international organisations.
(\today)
\end{english}
\bigskip

\begin{german}
\textbf{Die deutsche Sprache}\footnote{ From \url{http://de.wikipedia.org/wiki/Deutsche_Sprache}} (auch das Deutsche) gehört zum westlichen Zweig der germanischen Sprachen und ist eine der meistgesprochenen europäischen Sprachen weltweit, und gilt so als Weltsprache.
(\today)
\end{german}
\bigskip

\begin{russian}
\textbf{Русский язык} — один из восточнославянских языков, один из крупнейших языков мира, в том числе самый распространённый из славянских языков и самый распространённый язык Европы, как географически, так и по числу носителей языка как родного (хотя значительная, и географически бо́льшая, часть русского языкового ареала находится в Азии).	
(\today)
\end{russian}
\bigskip

\begin{latin}
\textbf{Lingua Latina} est lingua Indoeuropaea. Nomen ductum est de terra in paeninsula Italica quam Latine loquentes incolebant, Vetus Latium appellata sitaque inter flumen Tiberis, Volscam terram, mare Tyrrhenicum, montes Apenninos.

Quamquam sermone nativo fungi desinit, cumque nostris diebus perpauci Latine loqui possint, lingua mortua appellari solet, multas tamen peperit linguas quae linguae romanicae vocantur, sicut Hispanicam, Francogallicam, Italicam, Lusitanam, Dacoromanicam, Gallaicam, ne omnes afferam.
(\today) 

%Cum Antiqua Res Publica Romana et Imperium Romanum lingua Latina in rebus publicis privatisque usi sint, etiam linguam Graecam magnopere afficit, nec minus haec illam. Sicut lingua Graeca, Latina est lingua flexiva, ut nexus verborum non ex ordine appareat, ut mos est linguarum Romanicarum aut Anglicae, sed ex affixis.	
%
%% Cicero: Oratio Pro Marcello
%Diuturni silenti, patres conscripti, quo eram his temporibus usus, non
%timore aliquo, sed partim dolore, partim verecundia, finem hodiernus dies
%attulit, idemque initium quae vellem quaeque sentirem meo pristino more
%dicendi. Tantam enim mansuetudinem, tam inusitatam inauditamque clementiam,
%tantum in summa potestate rerum omnium modum, tam denique incredibilem
%sapientiam ac paene divinam tacitus praeterire nullo modo possum.  M. enim
%Marcello vobis, patres conscripti, reique publicae reddito non illius solum sed
%etiam meam vocem et auctoritatem vobis et rei publicae conservatam ac
%restitutam puto. Dolebam enim, patres conscripti, et vehementer angebar, cum
%viderem virum talem, cum in eadem causa in qua ego fuisset, non in eadem esse
%fortuna, nec mihi persuadere poteram nec fas esse ducebam versari me in nostro
%vetere curriculo illo aemulo atque imitatore studiorum ac laborum meorum quasi
%quodam socio a me et comite distracto. Ergo et mihi meae pristinae vitae
%consuetudinem, C. Caesar, interclusam aperuisti et his omnibus ad bene de re
%publica sperandum quasi signum aliquod sustulisti.  Intellectum est enim
%mihi quidem in multis et maxime in me ipso, sed paulo ante omnibus, cum M.
%Marcellum senatui reique publicae concessisti, commemoratis praesertim
%offensionibus, te auctoritatem huius ordinis dignitatemque rei publicae tuis
%vel doloribus vel suspicionibus anteferre. Ille quidem fructum omnis ante
%actae vitae hodierno die maximum cepit, cum summo consensu senatus tum iudicio
%tuo gravissimo et maximo. Ex quo profecto intellegis quanta in dato beneficio
%sit laus, cum in accepto sit tanta gloria. Est vero fortunatus cuius ex
%salute non minor paene ad omnis quam ad illum ventura sit laetitia pervenerit:
%quod quidem merito atque optimo iure contigit. Quis enim est illo aut
%nobilitate aut probitate aut optimarum artium studio aut innocentia aut ullo in
%laudis genere praestantior? 
\end{latin}
\bigskip

\begin{greek}
\textbf{Η ελληνική γλώσσα} είναι μία από τις ινδοευρωπαϊκές γλώσσες, για την
οποία έχουμε γραπτά κείμενα από τον 15ο αιώνα π.Χ. μέχρι σήμερα. Αποτελεί το
μοναδικό μέλος ενός κλάδου της ινδοευρωπαϊκής οικογένειας γλωσσών. Ανήκει
επίσης στον βαλκανικό γλωσσικό δεσμό.	
(\today) 
\end{greek}
\bigskip

\begin{quote}
\begin{greek}[variant=ancient]
τὸν δ' ἠμείβετ' ἔπειτα θεά, γλαυκῶπις Ἀθήνη:
“ὦ πάτερ ἡμέτερε Κρονίδη, ὕπατε κρειόντων,
καὶ λίην κεῖνός γε ἐοικότι κεῖται ὀλέθρῳ:
ὡς ἀπόλοιτο καὶ ἄλλος, ὅτις τοιαῦτά γε ῥέζοι:
ἀλλά μοι ἀμφ' Ὀδυσῆι δαί̈φρονι δαίεται ἦτορ,
δυσμόρῳ, ὃς δὴ δηθὰ φίλων ἄπο πήματα πάσχει
νήσῳ ἐν ἀμφιρύτῃ, ὅθι τ' ὀμφαλός ἐστι θαλάσσης.
νῆσος δενδρήεσσα, θεὰ δ' ἐν δώματα ναίει,
Ἄτλαντος θυγάτηρ ὀλοόφρονος, ὅς τε θαλάσσης
πάσης βένθεα οἶδεν, ἔχει δέ τε κίονας αὐτὸς
μακράς, αἳ γαῖάν τε καὶ οὐρανὸν ἀμφὶς ἔχουσιν.
τοῦ θυγάτηρ δύστηνον ὀδυρόμενον κατερύκει,
αἰεὶ δὲ μαλακοῖσι καὶ αἱμυλίοισι λόγοισιν
θέλγει, ὅπως Ἰθάκης ἐπιλήσεται: αὐτὰρ Ὀδυσσεύς,
ἱέμενος καὶ καπνὸν ἀποθρῴσκοντα νοῆσαι
ἧς γαίης, θανέειν ἱμείρεται. οὐδέ νυ σοί περ
ἐντρέπεται φίλον ἦτορ, Ὀλύμπιε. οὔ νύ τ' Ὀδυσσεὺς
Ἀργείων παρὰ νηυσὶ χαρίζετο ἱερὰ ῥέζων
Τροίῃ ἐν εὐρείῃ; τί νύ οἱ τόσον ὠδύσαο, Ζεῦ;”
(\today)
\end{greek}
\end{quote}
\bigskip

\begin{turkish}
\textbf{Türkiye Türkçesi}, Ural-Altay Dilleri içerisinde Türk dil ailesinin Oğuz Grubu'na mensup lehçedir. Anadolu, Kıbrıs, Balkanlar ve Orta Avrupa'da geniş yayılım alanı bulmuş olup, Türkiye Cumhuriyeti, Kuzey Kıbrıs Türk Cumhuriyeti, Güney Kıbrıs Rum Kesimi, Makedonya ve Kosova'nın resmî dilidir.
(Bugün: \today!)
\end{turkish}
\bigskip

\begin{polish}
\textbf{Język polski (polszczyzna)} należy wraz z językiem czeskim, słowackim, pomorskim (kaszubskim), dolnołużyckim, górnołużyckim oraz wymarłym połabskim do grupy języków zachodniosłowiańskich, stanowiących część rodziny języków indoeuropejskich. Ocenia się, że język polski jest językiem ojczystym około 44 milionów ludzi na świecie (w literaturze naukowej można spotkać szacunki od 40 do 48 milionów), mieszkańców Polski oraz Polaków zamieszkałych za granicą (Polonia).
(\today)
\end{polish}
\bigskip

\begin{latvian} 
\textbf{Latviešu valoda} ir dzimtā valoda apmēram 1,5 miljoniem cilvēku, galvenokārt Latvijā, kurā tā ir vienīgā valsts valoda. Lielākās latviešu valodas pratēju kopienas ārzemēs ir Austrālijā, ASV, Zviedrijā, Lielbritānijā, Vācijā, Brazīlijā, Krievijā. Latviešu valoda pieder indoeiropiešu valodu saimes baltu valodu grupai.
(\today)
\end{latvian}
\end{document}
