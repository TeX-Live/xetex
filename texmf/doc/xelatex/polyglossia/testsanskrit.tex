\documentclass[a4paper,12pt]{article}
\usepackage{fontspec}
\usepackage{polyglossia}
\setdefaultlanguage{french}
\setotherlanguage{sanskrit}
\setmainfont{Charis SIL}
\newfontfamily\sanskritfont[Script=Devanagari]{Arial Unicode MS}%{Mangal}

\catcode"00A0=\active
\def^^^^00a0{\char32\relax}

\title{Mahābhārata 1.3.1-9}
\author{}
\date{}

\begin{document}
\maketitle

Texte en français.

\begin{sanskrit}
जनमेजयः पारिक्षितः सह भ्रातृभिः कुरुक्षेत्रे दीर्घसत्रमुपास्ते । तस्य भ्रातरस्त्रयः श्रुतसेन उग्रसेनो भीमसेन इति । तेषु तत्सत्रमुपासीनेषु तत्र श्वाभ्यागच्छत्सारमेयः । स जनमेजयस्य भ्रातृभिरभिहतो रोरूयमाणो मातुः समीपमुपागच्छत् । तं माता रोरूयमाणमुवाच । किं रोदिषि । केनास्यभिहत इति । स एवमुक्तो मातरं प्रत्युवाच । जनमेजयस्य भ्रातृभिरभिहतोऽस्मीति । तं माता प्रत्युवाच । व्यक्तं त्वया तत्रापराद्धं येनास्यभिहत इति । स तां पुनरुवाच । नापराध्यामि किंचित् । नावेक्षे हवींषि नावलिह इति । तच्छ्रुत्वा तस्य माता सरमा पुत्रशोकार्ता तत्सत्रमुपागच्छद्यत्र स जनमेजयः सह भ्रातृभिर्दीर्घसत्रमुपास्ते । स तया क्रुद्धया तत्रोक्तः । अयं मे पुत्रो न किंचिदपराध्यति । किमर्थमभिहत इति । यस्माच्चायमभिहतोऽनपकारी तस्माददृष्टं त्वां भयमागमिष्यतीति । स जनमेजय एवमुक्तो देवशुन्या सरमया दृढं संभ्रान्तो विषण्णश्चासीत् ।
\end{sanskrit}
Texte en français.

\end{document}
