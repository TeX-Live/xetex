\part{Producing Presentations}
At the moment, there are two classes that you can prepare your presentations with. These classes are:
\section{xepersian-presentations class}
\textsf{xepersian-presentation} is a simple class for presentations to be shown on screen or beamer. It is derived from \LaTeX 's
article class.The “virtual paper size” of documents produced by this class: width=128mm,
height=96mm. \textsf{xepersian-presentation} requires that the \textsf{fancyhdr} and \textsf{geometry} packages are available on the system.
Enhancements to the \textsf{xepersian-presentation} class are easily made available by other packages, these include  slides with a background from a bitmap (\textsf{eso-pic}
package).

\subsection{Usage}
The class is used with
\begingroup
\catcode`\<=12
  \Mac \documentclass`[<options>]'{xepersian-presentation}
\endgroup

\File{OPTION!}
Options of the \textsf{article} class are also available to \textsf{xepersian-presentation}, e. g. "10pt", "11pt", "12pt" for selection of
font size. However, not all options
of the \textsf{article} class will be appropriate for a presentation class, e. g. "twocolumn".

\File{EXAMPLE}
A simple example document:
\begin{LVerb}
  \documentclass[12pt]{xepersian-presentation}
  \usepackage{eso-pic}
  \usepackage{xepersian}
  \settextfont[Scale=1]{XB Zar}
  \setromantextfont[Scale=1]{Linux Libertine}
  \setdigitfont[Scale=1]{XB Zar}
  \pagestyle{pres}
  \AddToShipoutPicture{
  \includegraphics{gradient2.png}
  }
  \begin{document}
  \begin{titlepage}
  \centering
  \distance{1}
  {
  \Huge \bfseries Title of the presentation \par
  }
  \vspace{1.3ex} \large
  Author\\[2ex]Institution
  \distance{2}
  \end{titlepage}
  \begin{plainslide}[Title of Page]
  The first page
  \end{plainslide}
  \begin{rawslide}
  The second page
  \end{rawslide}
  \end{document}
  \end{LVerb}

The title page can be created within the "titlepage" environment, the "\maketitle" command is not
available. Slides may be created with the "plainslide" environment, you may add the title of the slide with the
optional parameter. The contents of the slide are centered vertically.
Another environment generating a slide is "rawslide": slides are written without title, contents are not vertically
centered.

The \n\distance"{number}" command allows to introduce vertical space into slides constructed with the
"rawslide" and "titlepage" environments. You should use pairs of "\distance{}" commands with numbers
indicating the relative height of empty space, see the "titlepage" in the example above.

Pictures can be included with the "includegraphics"-command of the "graphicx"-package. Please be
aware that the dimensions of the pages are 128mm $\times$ 96mm and therefore included graphics are scaled
appropriately.
\subsection{Enhancements to xepersian-presentation}
\subsubsection{Fill background of a presentation with bitmaps}
\textsf{eso-pic} package  allows you to paint the background with a picture:
\begin{Ex}
"\usepackage{eso-pic}"\\
...\\
"\AddToShipoutPicture{"\\
"\includegraphics{gradient2.png}"\\
"}"
\end{Ex}

\n\AddToShipoutPicture"{}" puts the picture on every page, \n\AddToShipoutPicture"*{}" puts
it on to the current page, \n\ClearShipoutPicture\ clears the background beginning with the current
page. Details of "eso-pic"'s commands can be found in  its own documentation.

\section{beamer Class}
\XePersian\ currently works with \textsf{beamer} class quite well and all should go well with \textsf{default}, \textsf{Warsaw}, \textsf{Goettingen}, \textsf{Hannover}, \textsf{Malmoe}, \textsf{Marburg}, \textsf{Montpellier}, \textsf{PaloAlto}, \textsf{Pittsburgh}, \textsf{Singapore}, \textsf{Szeged} and \textsf{xepersian-JLTree}\footnote{\textsf{xepersian-JLTree} theme is the modified version of \textsf{JLTree} theme.} themes but other themes are not yet supported.

\begin{Warning}
I have defined some new environments for beamer and you should use these new  environments instead beamer's original environments. These new environments are:
\end{Warning}
\begin{Ex}
"\begin{"\textbf{\PersianTxt{قضیه}}"}"
\ldots\ "<text>" \ldots\
"\end{"\textbf{\PersianTxt{قضیه}}"}"
\end{Ex}
You should use \textbf{\PersianTxt{قضیه}} environment instead \textbf{theorem} environment.
\begin{Ex}
"\begin{"\textbf{\PersianTxt{نتیجه}}"}"
\ldots\ "<text>" \ldots\
"\end{"\textbf{\PersianTxt{نتیجه}}"}"
\end{Ex}
You should use \textbf{\PersianTxt{نتیجه}} environment instead \textbf{corollary} environment.
\begin{Ex}
"\begin{"\textbf{\PersianTxt{حقیقت}}"}"
\ldots\ "<text>" \ldots\
"\end{"\textbf{\PersianTxt{حقیقت}}"}"
\end{Ex}
You should use \textbf{\PersianTxt{حقیقت}} environment instead \textbf{fact} environment.
\begin{Ex}
"\begin{"\textbf{\PersianTxt{لما}}"}"
\ldots\ "<text>" \ldots\
"\end{"\textbf{\PersianTxt{لما}}"}"
\end{Ex}
You should use \textbf{\PersianTxt{لما}} environment instead \textbf{lema} environment.
\begin{Ex}
"\begin{"\textbf{\PersianTxt{مسئله}}"}"
\ldots\ "<text>" \ldots\
"\end{"\textbf{\PersianTxt{مسئله}}"}"
\end{Ex}
You should use \textbf{\PersianTxt{مسئله}} environment instead \textbf{problem} environment.
\begin{Ex}
"\begin{"\textbf{\PersianTxt{پاسخ}}"}"
\ldots\ "<text>" \ldots\
"\end{"\textbf{\PersianTxt{پاسخ}}"}"
\end{Ex}
You should use \textbf{\PersianTxt{پاسخ}} environment instead \textbf{solution} environment.
\begin{Ex}
"\begin{"\textbf{\PersianTxt{تعریف}}"}"
\ldots\ "<text>" \ldots\
"\end{"\textbf{\PersianTxt{تعریف}}"}"
\end{Ex}
You should use \textbf{\PersianTxt{تعریف}} environment instead \textbf{definition} environment.
\begin{Ex}
"\begin{"\textbf{\PersianTxt{تعریفها}}"}"
\ldots\ "<text>" \ldots\
"\end{"\textbf{\PersianTxt{تعریفها}}"}"
\end{Ex}
You should use \textbf{\PersianTxt{تعریفها}} environment instead \textbf{definitions} environment.
\begin{Ex}
"\begin{"\textbf{\PersianTxt{مثال}}"}"
\ldots\ "<text>" \ldots\
"\end{"\textbf{\PersianTxt{مثال}}"}"
\end{Ex}
You should use \textbf{\PersianTxt{مثال}} environment instead \textbf{example} environment.
\begin{Ex}
"\begin{"\textbf{\PersianTxt{مثالها}}"}"
\ldots\ "<text>" \ldots\
"\end{"\textbf{\PersianTxt{مثالها}}"}"
\end{Ex}
You should use \textbf{\PersianTxt{مثالها}} environment instead \textbf{examples} environment.

\begin{Ex}
"\begin{"\textbf{\PersianTxt{اثبات}}"}"
\ldots\ "<text>" \ldots\
"\end{"\textbf{\PersianTxt{اثبات}}"}"
\end{Ex}
You should use \textbf{\PersianTxt{اثبات}} environment instead \textbf{proof} environment.
\begin{Ex}
"\begin{"\textbf{\PersianTxt{بلوک}}"}{<title>}"
\ldots\ "<text>" \ldots\
"\end{"\textbf{\PersianTxt{بلوک}}"}"
\end{Ex}
You should use \textbf{\PersianTxt{بلوک}} environment instead \textbf{block} environment.
\begin{Ex}
"\begin{"\textbf{\PersianTxt{بلوک‌مثال}}"}{<title>}"
\ldots\ "<text>" \ldots\
"\end{"\textbf{\PersianTxt{بلوک‌مثال}}"}"
\end{Ex}
You should use \textbf{\PersianTxt{بلوک‌مثال}} environment instead \textbf{exampleblock} environment.
\begin{Ex}
"\begin{"\textbf{\PersianTxt{بلوک‌هشدار}}"}{<title>}"
\ldots\ "<text>" \ldots\
"\end{"\textbf{\PersianTxt{بلوک‌هشدار}}"}"
\end{Ex}
You should use \textbf{\PersianTxt{بلوک‌هشدار}} environment instead \textbf{alertblock} environment.

\subsection{An Example}

\File{Example}
\begin{LVerb}
  \documentclass{beamer}
  \usetheme{Warsaw}
  \usepackage{xepersian}
  \settextfont[Scale=1]{XB Zar}
  \setromantextfont[Scale=1]{Linux Libertine}
  \AtBeginDocument{\setdigitfont[Scale=1]{XB Zar}}
  \title{<title>}
  \subtitle{<subtitle>}
  \author{<author>}
  \institute{<institute>}
  \begin{document}
  \begin{frame}
  \titlepage
  \end{frame}
  \begin{frame}{<frame-title>}
  \framesubtitle{<frame-subtitle>}
  ... <text> ...
  \end{frame}
  ...
  \end{document}
\end{LVerb}
\endinput
