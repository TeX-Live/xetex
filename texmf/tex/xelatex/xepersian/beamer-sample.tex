\documentclass{beamer}
\usetheme{Warsaw}
\usepackage{xepersian}
\settextfont[Scale=1]{XB Zar}
\setromantextfont[Scale=1]{Linux Libertine}
\AtBeginDocument{\setdigitfont[Scale=1]{XB Zar}}
\title{عنوان اسلاید}
\subtitle{این یک زیرعنوان است}
\author{وفا خلیقی}
\institute{دانشگاه تهران}
\begin{document}
\begin{frame}
\titlepage
\end{frame}


\begin{frame}{این یک عنوان است}
\framesubtitle{این زیرعنوان است}
\begin{مثال}
در این مثال به برسی توابعی خواهیم پرداخت که برای ما قابل استفاده هستند:
\begin{equation}
(x-y)^2=x^2-2xy+y^2
\end{equation}
\end{مثال}
\begin{پاسخ}
این یک پاسخ است
\end{پاسخ}
\begin{مسئله}
این یک مسئله است.
\end{مسئله}
\end{frame}
\begin{frame}{ادامهّ عنوان قبلی}
\framesubtitle{ادامهٔ زیرعنوان قبلی}
\begin{قضیه}
این یک قضیه است.
\end{قضیه}
\begin{اثبات}
این یک اثبات است
\end{اثبات}
\begin{نتیجه}
این یک نتیجه است
\end{نتیجه}
\begin{حقیقت}
این یک حقیقت است
\end{حقیقت}
\end{frame}
\begin{frame}{ادامهّ عنوان قبلی}
\framesubtitle{ادامهٔ زیرعنوان قبلی}
\begin{لم}
این یک لم است
\end{لم}
\begin{تعریف}
این یک تعریف است
\end{تعریف}
\begin{تعریفها}
این محیط مخصوص بیشتر از یک تعریف است
\end{تعریفها}
\begin{مثالها}
این محیط مخصوص بیشتر از یک مثال است.
\end{مثالها}
\end{frame}
\begin{frame}{ادامهّ عنوان قبلی}
\framesubtitle{ادامهٔ زیرعنوان قبلی}
\begin{بلوک}{محیط بلوک}
این محیط بلوک است
\end{بلوک}
\begin{بلوک‌مثال}{محیط بلوک مثال}
این محیط بلوک مثال است
\end{بلوک‌مثال}
\begin{بلوک‌هشدار}{محیط بلوک هشدار}
این محیط بلوک هشدار است.
\end{بلوک‌هشدار}
\end{frame}
\begin{frame}{ادامهّ عنوان قبلی}
\framesubtitle{ادامهٔ زیرعنوان قبلی}
این هم مقداری متن معمولی می‌نویسم تا بتوانم به خط بعدی بروم و ببینم که آیا واقعاً \lr{beamer} دست از لجاجت برمی‌دارد یا نه
\begin{roman}
This is an English paragraph that I am writing here to see what happens after this, will I get a new bug or I am lucky enough?
\end{roman}
این دوباره متنی است فارسی که من آن را می‌نویسم و هنوز در حال آزمایش هستم تا بتوانم به خط بعدی بروم و نتیجه را مشاهده کنم.
\end{frame}
\end{document}
