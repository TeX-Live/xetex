\documentclass[11pt,a4paper]{article} 
% محمود امین‌طوسی، http://webpages.iust.ac.ir/mamintoosi

\usepackage{verbatim}
\usepackage{pstricks}
\usepackage{animate}
\usepackage{pstricks-add}
\usepackage{multido}
\usepackage{color}
\usepackage{fancybox}
\usepackage{fancyvrb}
\usepackage{setspace}
\usepackage[top=35mm, bottom=32mm, left=28mm, right=28mm]{geometry}

\usepackage{xepersian}
\settextfont[Scale=1]{XB Niloofar}
\setromantextfont[Scale=1]{Linux Libertine}%{Arial}%
\setdigitfont{Parsi Digits}
\defpersianfont\Andalus[Scale=1]{Andalus}
\defpersianfont\Sayeh[Scale=1]{XB Kayhan Sayeh}
\defromanfont\Courier[Scale=1]{Courier New Bold}

\theoremstyle{plain} \newtheorem{question}{پرسش}%[section]
\newcommand{\answer}{{\noindent \Sayeh پاسخ: }}
\newcommand\SLASH{\char`\\}
\newcommand{\textblue}[1]{{\addfontfeature{Color=0000FF}#1}}

\title{{\vspace{-20mm}
\Andalus بسم الله الرحمن الرحيم}
\psline[linewidth=19mm,linecolor=yellow](-90mm,0)(\linewidth,0)
\vspace{20mm}\\
{\shadowbox{\textblue{راهنمای استفاده از سبک‌های فارسی  برای \lr{\textcolor{red}{\textsc{Bib}\TeX}} در زی‌پرشین}}}
}
\author{محمود امین‌طوسی و مصطفی واحدی\\
گروه فارسی لاتک\\
\lr{http://wiki.parsilatex.org}\\
\lr{\{m.amintoosi,mostafa.vahedi\} at gmail.com}
}
\begin{document}
\maketitle

\tableofcontents

\section{مقدمه}
%{\shadowbox{\textblue{راهنمای استفاده از سبک‌های فارسی  برای }}}
برای متون لاتین انواع مختلفی از فایلهای سبک \lr{.bst} موجودند که هر یک مراجع را با قالب خاصی نمایش می‌دهند. به عنوان مثال \lr{IEEE} چند سبک مخصوص به خود دارد
که نویسندگان مقالات برای مجموعه‌ی \lr{IEEE} باید برای آماده‌سازی مراجع خود مطابق آن عمل نمایند. مزیت استفاده از این سبک‌ها آن است که هنگامی که با 
مجموعه‌ای از مراجع سروکار دارید که در هر مقاله به برخی از آنها ارجاع می‌دهید نیازی نیست که درگیر قالب مراجع و مرتب کردن آنها باشید. 
اگر کنفرانس یا مجله مورد نظر شما خودش فایل سبکی داشته باشد فقط کافیست فایل سبک آنرا داشته، نام آنرا در قسمت بیوگرافی قرار دهید، 
باقی کارها به صورت خودکار انجام خواهد شد. در استفاده از \lr{\textsc{Bib}\TeX} همانند سیستم تک، ممکن است کاربر در ابتدا کمی کارش زیادتر شود ولی در نهایت هم راحت‌تر بوده و هم 
سرعت کارش بیشتر خواهد بود.

هنگامی‌که افراد یک بانک از مراجع خودشان آماده نمایند خواهند توانست به راحتی یک یا چند ارجاع به مراجع را حذف یا اضافه ‌نمایند؛ 
مراجع به صورت خودکار مرتب شده و فقط مراجع ارجاع داده شده در قسمت بیوگرافی خواهندآمد. قالب مراجع به صورت یکدست مطابق سبک داده شده بوده و نیازی نیست
که کاربر درگیر قالب‌دهی به مراجع باشد. در ادامه سبک‌های فعلی فارسی قابل استفاده در زی‌پرشین و نحوه استفاده از آنها بیان خواهد شد.


\section{سبک‌های فعلی قابل استفاده در زی‌پرشین}
در حال حاضر فایلهای سبک زیر در نسخه‌ی آزمایشی برای استفاده در زی‌پرشین آماده شده‌اند:
\begin{description}
\item [unsrt-fa.bst] این سبک متناظر با \lr{unsrt.bst} می‌باشد. مراجع به ترتیب ارجاع در متن ظاهر می‌شوند.
\item [plain-fa.bst] این سبک متناظر با \lr{plain.bst} می‌باشد. مراجع بر اساس نام‌خانوادگی نویسندگان، به ترتیب صعودی مرتب می‌شوند.
 همچنین ابتدا مراجع فارسی و سپس مراجع انگلیسی خواهند آمد.
\item [acm-fa.bst] این سبک متناظر با \lr{acm.bst} می‌باشد. شبیه \lr{plain-fa.bst} است.  قالب مراجع کمی متفاوت است. اسامی نویسندگان انگلیسی با حروف بزرگ انگلیسی نمایش داده می‌شوند.
\item [ieeetr-fa.bst] این سبک متناظر با \lr{ieeetr.bst} می‌باشد. مراجع مرتب نمی‌شوند.
%\item [persia-unsorted.bst] این سبک  شبیه \lr{ieeetr-fa.bst} می‌باشد با این تفاوت که برخی نامها با حروف توپر نوشته شده‌اند.
\end{description}
برای مشاهده‌ی تفاوت خروجی این سبک‌ها به فایلهای \lr{bibtex\_example.pdf}، \lr{plain-fa-output.pdf}، \lr{acm-fa-output.pdf} و \lr{ieeetr-fa-output.pdf} که همراه 
با زی‌پرشین ارائه شده‌اند مراجعه فرمایید.

\section{ نحوه استفاده از سبک‌های فارسی}
 با مطالعه و اجرای مثال ارائه شده با زی‌پرشین (فایل \lr{bibtex\_example.tex})  با نحوه‌ی استفاده از سبک‌های فارسی آشنا خواهید شد. مراحل اصلی 
 برای استفاده از این سبک‌ها در ذیل آمده است:
\begin{enumerate}
\item در ابتدا باید یک بانک از مراجع خود همانند فایل \lr{MyReferences.bib} تهیه نمایید. اغلب انواع مراجع معمول مورد استفاده در آن آمده است. این فایل 
را در هر زمان می‌توانید ویرایش نموده، مراجعی را حذف یا اضافه نمایید. به علاوه فقط مراجعی از این فایل که در متن به آنها ارجاع داده شده باشد در خروجی 
سند شما قرار خواهد گرفت.
\item برای هر مدخل فارسی بایستی فیلدی با نام \lr{language} و با مقدار \lr{persian} داشته باشید.
\item در محلی از سورس زی‌پرشین خود که می‌خواهید لیست مراجع قرار بگیرد (معمولاً آخر سند) دستورات زیر را قرار دهید:
\Roman
\begin{verbatim}
\bibliographystyle{style-file-name}% such as plain-fa 
\bibliography{bib-file-name} %such as MyReferences
\end{verbatim}
\Persian
\item دنباله پردازشهای زیر را برای حصول به نتیجه نهایی انجام دهید:
\Roman
\begin{verbatim}
xelatex filename  
bibtex  filename  
xelatex filename  
xelatex filename  
\end{verbatim}
\Persian
\end{enumerate}

\section{یک فایل \lr{bib} شامل چیست؟}
یک فایل \lr{bib} در واقع یک پایگاه داده از مراجع\Footnote{Bibliography Database}  شماست که هر مرجع در آن به عنوان یک رکورد از این پایگاه داده
با قالبی خاص ذخیره می‌شود. به هر رکورد یک مدخل\Footnote{Entry} گفته می‌شود. یک نمونه مدخل برای معرفی کتاب \lr{Digital Image Processing} در ادامه آمده است:
\Roman
\begin{verbatim}
@BOOK{Gonzalez02image,
  AUTHOR =      {Rafael Gonzalez and Richard Woods},
  TITLE =       {Digital Image Processing},
  PUBLISHER =   {Prentice-Hall, Inc.},
  YEAR =        {2006},
  ISBN = 	      {013168728X},
  EDITION =     {3rd},
  ADDRESS =     {Upper Saddle River, NJ, USA}
}
\end{verbatim}
\Persian
در مثال فوق، \lr{@BOOK} مشخصه‌ی شروع یک مدخل مربوط به یک کتاب و \lr{Gonzalez02book} برچسبی است که به این مرجع منتسب شده است.
 این برچسب بایستی یکتا باشد. برای آنکه فرد به راحتی بتواند برچسب
مراجع خود را به خاطر بسپارد و حتی‌الامکان برچسب‌ها متفاوت با هم باشند معمولاً از قوانین خاصی به این منظور استفاده می‌شود. یک قانون می‌تواند فامیل نویسنده‌ی
اول+دورقم سال نشر+اولین کلمه‌ی عنوان اثر باشد. به \lr{AUTHOR} و $\dots$ و \lr{ADDRESS} فیلدهای این مدخل گفته می‌شود؛ که هر یک با مقادیر مربوط
به مرجع مقدار گرفته‌اند. ترتیب فیلدها مهم نیست. 

انواع متنوعی از مدخل‌ها برای اقسام مختلف مراجع همچون کتاب، مقاله‌ی کنفرانس و مقاله‌ی ژورنال وجود دارد که برخی فیلدهای آنها با هم متفاوت است. 
نام فیلدها بیانگر نوع اطلاعات آن می‌باشد. مثالهای ذکر شده در فایل \lr{MyReferences.bib} کمک خوبی به شما خواهد بود. 
این فایل یک فایل متنی بوده و با ویرایشگرهای معمول همچون \lr{Notepad++} قابل ویرایش می‌باشد. برنامه‌هایی همچون 
\lr{TeXMaker} امکاناتی برای نوشتن این مدخل‌ها دارند و به صورت خودکار فیلدهای مربوطه را در فایل \lr{bib}  شما قرار می‌دهند.  
با استفاده از سبک‌های فارسی آماده شده، محتویات هر فیلد می‌تواند به فارسی نوشته شود، ترتیب مراجع و نحوه‌ی چینش فیلدهای هر مرجع را سبک مورد استفاده 
مشخص خواهد کرد.

برای اطلاع از انواع فایلهای \lr{bst} و نحوه‌ی کار با \lr{\textsc{Bib}\TeX} به سایت‌های زیر مراجعه فرمایید:
\begin{roman}
\noindent http://www.tex.ac.uk/cgi-bin/texfaq2html/\\
\noindent http://oldsite.maths.uwa.edu.au/computing/software/tex/doc/texhelp/bibtex-c.html\\
\noindent http://www.bibtex.org/
\end{roman}

\section{چند نکته در مورد استفاده از سبک‌های فارسی}
\begin{itemize}
\item در حال حاضر نام کوچک نویسندگان فارسی به صورت کامل نمایش داده می‌شود. در صورت نیاز به ذکر فقط حرف اول نام، مدخل مراجع خود را اصلاح فرمایید.
\item به جای ایجاد فایل \lr{bib} جدید برای خود،  فایل \lr{MyReferencesbib} که با رمزینه‌ی \lr{UTF-8} ذخیره شده است را تغییر داده در صورت نیاز با نامی دیگر ذخیره نمایید. 
در غیراینصورت با مشکل مواجه خواهید شد.
\item برای فونت فارسی از فونت‌های سری \lr{XB} استفاده نمایید.
\item در حال حاضر در تمامی سبک‌های فارسی ارائه شده (برخلاف فرم اصلی برخی از آنها) ابتدا نام خانوادگی و سپس نام افراد ظاهر می‌‌شود.
\item ممکن است خروجی فایل‌های سبک برای مراجع انگلیسی کمی متفاوت با متناظر اصلی آنها باشد. برای متون صرفاً لاتین خود از سبک‌های اصلی استفاده نمایید.
\item برچسب‌ هر مرجع می‌تواند به فارسی نوشته شود، لیکن در آن نباید فاصله بکار برده شده باشد. 
به عنوان مثال به جای 'امین طوسی` بایستی از نیم فاصله استفاده نمود و آنرا به صورت 'امین‌طوسی` نوشت.
یک قانون خوب می‌تواند استفاده از برچسب فارسی برای مراجع فارسی و برچسب انگلیسی برای مراجع انگلیسی باشد.
\item سبک‌های فارسی در نسخه‌ آزمایشی هستند. برای گزارش مشکل به تالار گفتگوی فارسی‌لاتک  در آدرس زیر مراجعه فرمایید:\hfill \lr{http://forum.parsilatex.org}
\end{itemize}

\section{مرتب‌سازی مراجع بر اساس نام نویسنده}
در حال حاضر مرتب بودن مراجع در سبک‌های \lr{plain-fa} و \lr{acm-fa} نیاز به کمی کار دستی دارد:
\begin{enumerate}
\item مطابق فایل \lr{MyReferences.bib} برای استفاده از امکان مرتب‌سازی مراجع، نام نویسنده‌ی اول در فیلد \lr{AUTHOR} را 
در دستور \lr{noopsort} قرار دهید. محتویات اولین آکولاد در خروجی چاپ نشده و 
صرفاً در مرتب‌سازی مورد استفاده قرار می‌گیرد. اگر نیازی به مرتب‌سازی ندارید استفاده از \lr{noopsort} الزامی نیست.
\item اگر در نام خانوادگی نویسنده‌ی اول یکی از حروف 'گ`، 'چ`، 'پ` و 'ژ` باشد مرتب‌سازی با مشکل مواجه خواهد شد. برای رفع مشکل، در اولین آکولاد پس 
از دستور \lr{noopsort} به قبل از هر یک از این حروف، حرف ماقبل آن در ترتیب الفبای فارسی را اضافه نمایید. به عنوان مثال 'پورموسی` به صورت 'بپورموسی`
 در آکولاد اول نوشته خواهد شد. این نامِ تغییر یافته فقط جهت مرتب‌سازی مورد استفاده قرار خواهد گرفت و در خروجی ظاهر نخواهد شد.
\end{enumerate}
سعی بر آن است که در نسخه‌های آتی این مشکل برطرف شده و عملیات به صورت خودکار انجام شود.
\section{پرسش و پاسخ} 
\begin{question} آیا می‌توان شماره صفحات ارجاعی به هر مرجع را در انتهای هر مرجع داشت؟ این حالت مخصوصاً هنگام‌ داوری یک مقاله یا پروژه 
خیلی مفید است.
\end{question}
\answer
بله، با استفاده از بسته‌ی \lr{backref} می‌توان این کار را انجام داد. به این منظور کافیست دستور \lr{\textbackslash usepackage\{backref\}} را در 
قسمت آغازین سند خود قرار دهید.
%این بسته را می‌توان به تنهایی یا همراه با بسته‌ی \lr{hyperref} مورد استفاده قرار داد. در حال حاضر 
البته اگر به همین ترتیب استفاده شود، پس از هر مرجع شماره‌ی صفحات ارجاعی به آن پس از کلمه‌ی \lr{pages} خواهد آمد که برای تبدیل آن به معادل فارسی 
آن یعنی 'صفحات` کافیست دستور زیر را پس از دستور فوق قرار دهید:
%\lr{\}\}}صفحات\lr{\SLASH def\SLASH backrefpagesname\{}
\begin{roman}
\noindent{ \SLASH def\SLASH backrefpagesname\{\rl{صفحات}\}}
\end{roman}
 اگر تمام مراجع شما فارسی باشند مشکلی نخواهد بود لیکن در صورت
داشتن مراجع انگلیسی این مسئله مشکل‌دار خواهد بود. تا زمانی که این بسته برای زی‌پرشین سازگار نشده است می‌توانید آنرا به صورت زیر بکار برید:
\begin{roman}
\begin{verbatim}
\usepackage{backref}
\def\backrefpagesname{}
\end{verbatim}
\end{roman}
البته به این ترتیب فقط شماره صفحات ارجاعی را پس از هر مرجع خواهید داشت.
\begin{question} گاهی اوقات اخطار \lr{Underfull \SLASH hbox} را دریافت می‌کنیم. مشکل از چیست؟ \end{question}
\answer در برخی حالات به دلیل عدم توانایی تک در تنظیم بهینه‌ی محل شکستن خطوط این اخطار داده می‌شود که مهم نیست.

\begin{question} چرا به جای خط فاصله در بین شماره صفحات مراجع فارسی یک مربع چاپ می‌شود؟ \end{question}
\answer شما از فونت‌ مناسبی برای فارسی استفاده نکرده‌اید. از فونت‌های سری \lr{XB} استفاده نمایید. برای آدرس دانلود به ویکی فارسی‌لاتک مراجعه نمایید.

\begin{question} چرا در سبک \lr{ieeetr-fa}  شماره‌ی مجله (\lr{number}) در مراجع از نوع \lr{article} نمایش داده نمی‌شود؟ \end{question}
\answer در سبک اصلی \lr{ieeetr} اگر ماه نشر مجله مشخص باشد، شماره‌ی آن نمایش داده نخواهد شد.
 همچنین دقت داشته باشید که این سبک کاملاً مطابق با سبک جدید \lr{IEEEtran} نیست.

\begin{question} حروف فارسی در لیست مراجع من به‌هم ریخته و ناخوانا است. مشکل از چیست؟ \end{question}
\answer همان‌گونه که قبلاً اشاره شد فایل \lr{bib} بایستی با رمزینه‌ی \lr{UTF-8} ذخیره شده باشد. به جای ایجاد فایل \lr{bib} جدید برای خود، 
فایل \lr{MyReferencesbib} را که با رمزینه‌ی \lr{UTF-8} ذخیره شده است را تغییر داده در صورت نیاز با نامی دیگر ذخیره نمایید. 

\end{document}