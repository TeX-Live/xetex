\documentclass[11pt,a4paper]{article} 
% محمود امین‌طوسی، http://webpages.iust.ac.ir/mamintoosi
% فایل حاضر مثالی برای استفاده از سبک‌های فارسی در زی‌پرشین می‌باشد.
% توجه نمایید که برای استفاده از این مثال باید به صورت زیر عمل نمایید:
% xelatex filename  (در نوت‌پد++ F6 یا Ctrl+F6) - با توجه به راهنمای زی‌پرشین
% bibtex  filename  (تایپ دستور در خط فرمان نوت‌پد++)
% xelatex filename  (در نوت‌پد++ F6)
% xelatex filename  (در نوت‌پد++ F6 یا Ctrl+F6)
% در این مثال به جای filename باید bibtex_example قرار دهید.
% برای ویرایش مدخل‌ها فایل MyReferences.bib را باز کنید. در صورت تغییر این فایل بایستی دو مرحله‌ی میانی را پس از ذخیره‌ی آن اجرا نمایید.
% در صورت تغییر در ارجاعات در این فایل باید هر چهار مرحله انجام شوند.

% به جای unsrt می‌توانید سایر سبک‌ها را قرار داده و نتیجه را مشاهده نمایید. (plain-fa, acm-fa, ieeetr-fa, persia)

\usepackage{verbatim}
\usepackage{xepersian}
\settextfont[Scale=1]{XB Zar}
\setromantextfont[Scale=.95]{Linux Libertine}%{Times New Roman}
\setdigitfont{Parsi Digits}

\title{نمونه خروجی با استیل فارسی \lr{unsrt-fa} برای \lr{BibTeX} در زی‌پرشین}
\author{}\date{}
\begin{document}
\maketitle

مرجع \cite{امیدعلی82دکترا} یک نمونه پروژه دکترا و مرجع\cite{واحدی87} یک نمونه مقاله مجله فارسی است.
مرجع \cite{Baker02limits} یک نمونه مقاله انگلیسی است که در بین مراجع فارسی قرار گرفته است، مرجع \cite{Amintoosi87using}  یک نمونه  مقاله کنفرانس فارسی و
مرجع \cite{استالینگ۸۰کتاب} یک نمونه کتاب فارسی با ذکر مترجمان و ویراستاران فارسی است. مرجع \cite{Khalighi07MscThesis} یک نمونه پروژه کارشناسی ارشد انگلیسی و
\cite{خلیقی۸۷زی‌پرشین} هم یک نمونه متفرقه  می‌باشند.

%\nocite{*}
{\small
\bibliographystyle{unsrt-fa}%{plain-fa}%{ieeetr-fa}%{persia}%
\bibliography{MyReferences}
}

\end{document}