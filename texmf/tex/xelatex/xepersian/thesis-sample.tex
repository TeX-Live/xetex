\documentclass[a4paper,11pt]{xepersian-thesis}
\usepackage{xepersian}
\settextfont[Scale=1]{XB Zar}
\setromantextfont[Scale=1]{Linux Libertine}
\setdigitfont[Mapping=parsidigits]{XB Zar}




\begin{document}
\title{اعداد مرکب}
\author{وفا خلیقی}
\degree{کارشناسی ریاضیات محض}
\supervisor{محمد قدسی}
\department{ریاضی}
\university{تهران}
\city{تهران}
\thesisdate{\today}
\maketitle
\begin{acknowledgementpage}
در اینجا دوست دارم از همه تشکر کنم
\end{acknowledgementpage}
\begin{abstractpage}
اینجا هم باید مطلبی نوشت...
\keywords{ریاضی محض، هندسه و توپولوژی}
\end{abstractpage}
\tableofcontents
\listoftables
\chapter{مجموعه اعداد}
در اینجا لازم می‌دانم تا نگاهی روی مجموعهٔ اعداد اندازیم تا متوجه خصلت‌های آن شویم
\begin{equation}
(a+b)^2=a^2+2ab+b^3
\end{equation}
\newpage
این صفحهٔ جدید است که ما در این صفحه به نوشتن مقالات دیگری مشغول هستیم.
\section{مقدمه}

\newpage
این هم یک صفحهٔ جدید دیگر
\chapter{دوم}
\chapter{سوم}
\appendix
\chapter{اول}
\chapter{دوم}
\chapter{سوم}
\bibliographystyle{unsrt}
\begin{thebibliography}{99}
\end{thebibliography}
%\printindex
\begin{english}
\englishtitle{Pure Mathematics}
\englishauthor{Vafa Khalighi}
\englishdegree{Bachelor of Pure Mathematics}
\englishthesisdate{\englishtoday}
\englishsupervisor{Mohammad Ghodsi}
\englishdepartment{Computer Science}
\englishuniversity{Iran University of Science \& Technology}
\englishcity{Tehran}
\begin{englishabstract}
\noindent This is our abstract written in English and This thesis is written very very very carefully so that we can keep  an eye on the beauty of typesetting.
\englishkeywords{Mathematics, topology, geometry and category theory.}
\end{englishabstract}
\makeenglishtitle
\end{english}
\end{document}

