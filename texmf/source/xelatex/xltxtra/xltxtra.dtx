% \iffalse
%
%!TEX encoding = UTF-8 Unicode
%
% Copyright (C) 2006 by Will Robertson <wspr81@gmail.com>
% 
% Distributable under the LaTeX Project Public License,
% version 1.3c or higher (your choice). The latest version of
% this license is at: http://www.latex-project.org/lppl.txt
%
% This work is "maintained" (as per LPPL maintenance status) 
% by Will Robertson.
% 
% This work consists of the file  xltxtra.dtx
%           and the derived files xltxtra.sty and xltxtra.pdf.
%
%
%<*internalbatchfile>
\begingroup
%</internalbatchfile>
%<*batchfile>
\input docstrip.tex
\keepsilent
\preamble

  ____________________________
  The xltxtra package         
  (C) 2006  Will Robertson    
  License information appended


\endpreamble
\postamble

Copyright (C) 2006 by Will Robertson <wspr81@gmail.com>

Distributable under the LaTeX Project Public License,
version 1.3c or higher (your choice). The latest version of
this license is at: http://www.latex-project.org/lppl.txt

This work is "maintained" (as per LPPL maintenance status) 
by Will Robertson.

This work consists of the file  \jobname.dtx
          and the derived files \jobname.sty and \jobname.pdf.

\endpostamble
\askforoverwritefalse
\generate{\file{\jobname.sty}{\from{\jobname.dtx}{package}}}
\generate{\file{\jobname.ins}{\from{\jobname.dtx}{batchfile}}}
\nopreamble\nopostamble
\generate{\file{README.txt}{\from{\jobname.dtx}{readme}}}
\generate{\file{dtx-style.sty}{\from{\jobname.dtx}{dtx-style}}}
%</batchfile>
%<batchfile>\endbatchfile
%<*internalbatchfile>
\endgroup
%</internalbatchfile>
%
%<*driver>
\documentclass[12pt]{ltxdoc}
\EnableCrossrefs
\CodelineIndex
\RecordChanges
%\OnlyDescription
\usepackage{dtx-style}
\begin{document}
  \DocInput{\jobname.dtx}
\end{document}
%</driver>
%
%<*readme>
------------------------
THE XLTXTRA PACKAGE v0.2

This package provides a few small features that 
are useful for XeLaTeX.

It loads the fixltx2e and etex packages, and
patches a couple of other LaTeX commands: \-
for discretionary hyphenation with variable
hyphenation characters, and \textsuperscript/
\textsubscript that use the fontspec to access
real superior/inferior characters. It also
provides a parameterised logos for \TeX, 
\LaTeX and \XeTeX.

--------------
Will Robertson
2006
%</readme>
%
% \fi
%
% \errorcontextlines=999
% \makeatletter
%
% \GetFileInfo{\jobname.sty}
%
% \title{The \pkg{xltxtra} package}
% \author{Will Robertson}
% \date{\filedate \qquad \fileversion}
%
% \maketitle
% \tableofcontents
%
% \section{Introduction}
%
% This document describes the \pkg{xltxtra} package. It implements in general
% improved functionality for broken or sub-standard \LaTeX\ methods when using the
% \XeTeX\ format.
%
% Documentation is slim, and a bit ugly at this stage. I intend to improve it at some stage\dots Anyway, look through the sections to see what's contained within.
%
% \part{The \textsf{xltxtra} package}
%\iffalse
%<*package>
%\fi
% This is the package.
%    \begin{macrocode}
\ProvidesPackage{xltxtra}
  [2006/06/13 v0.2 Improvements for the XeTeX/LaTeX format]  
%    \end{macrocode}
%
% \paragraph{Required packages}
%    \begin{macrocode}
\RequirePackage{ifxetex}
\RequireXeTeX
\RequirePackage{graphicx}
\RequirePackage{fontspec}
\RequirePackage{xunicode}
%    \end{macrocode}
%
% \section{Logos}
% \DescribeMacro{\XeTeX}
% \DescribeMacro{\XeLaTeX}
% The \TeX-related logos people insist upon using need to be tuned
% on a per-font  basis. This package will eventually allow this,
% but for now, it's baby steps. The \XeTeX\ and \XeLaTeX\ logos
% are provided.
%
% The various \TeX-like logos that extend outside the regular
% vertical alphabetic bounds of running text have the unfortunate
% side-effects in \XeTeX\ of often overrunning the \cmd\baselineskip.
% Putting the logos in zero-height boxes prevents this problem.
% Actually, this problem doesn't happen anymore.
%
% Here're some examples. The default:
% \begin{example}
%   \TeX\ \XeTeX\ \LaTeX\ \XeLaTeX
% \end{example}
% Notice it's a bit tight when not using Computer Modern, as here:
% \begin{example}
%   \usefont{OT1}{cmr}{m}{n}
%   \TeX\ \XeTeX\ \LaTeX\ \XeLaTeX
% \end{example}
% This package provides a \emph{non-stable} method of specifying
% the spacings in these logos. In the future, it will hopefully adjust
% somewhat automatically.
%
% \noindent To do:
% \begin{itemize}[noitemsep]
% \item adapt \cmd\LaTeX\ to use small caps if available\dots
% \item \dots otherwise, need a scaling factor, and maybe a vertical nudge factor
% \item add other logos
% \item per-font parameters, with some defaults for common fonts
% \item add `low contrast' small caps versions, et al.
% \item probably break out the whole thing into its own package, if it works
% \end{itemize}
%
% \begin{macro}{\TeX@logo@spacing}
% \darg{Kern between T \& eX}
% \darg{Kern between Te \& X}
% \darg{Lowering amount for E in TeX}
% \darg{Kerning between L \& aTeX}
% \darg{Kerning between La \& TeX}
% \darg{Kerning between Xe \& LaTeX}
% This macro defines new \cmd\TeX\ and \cmd\XeTeX\ logos.
% Parameters must be tuned on a per-font basis:
% \begin{example}
%   \TeX@logo@spacing{-0.12em}{-0.12em}%
%     {0.5ex}{-0.3em}{-0.12em}{-0.1em}
%   \TeX\ \XeTeX\ \LaTeX\ \XeLaTeX
% \end{example}
%
% \emph{Warning!} This macro will  {\fontspec{Hoefler Text Black} definitely} change in the future.
% If you care abouts backwards compatibility in your documents, copy+paste
% the definitions below rather than using \cmd\TeX@logo@spacing.
%
%    \begin{macrocode}
\newlength\xxt@kern@Te
\newlength\xxt@kern@eX
\newlength\xxt@lower@e
\newlength\xxt@kern@La
\newlength\xxt@kern@aT
\newlength\xxt@kern@eL
\newcommand*\TeX@logo@spacing[6]{%
  \setlength\xxt@kern@Te{#1}%
  \setlength\xxt@kern@eX{#2}%
  \setlength\xxt@lower@e{#3}%
  \setlength\xxt@kern@La{#4}%
  \setlength\xxt@kern@aT{#5}%  
  \setlength\xxt@kern@eL{#6}%  
}
\DeclareRobustCommand\TeX{%
  \leavevmode
  \smash{%
    T\kern\xxt@kern@Te
    \lower\xxt@lower@e\hbox{E}\kern\xxt@kern@eX X}%
  \spacefactor1000\relax}
\DeclareRobustCommand{\LaTeX}{%
  \leavevmode
  \smash{%
  L\kern\xxt@kern@La 
  {\sbox\z@ T% 
    \vbox to\ht\z@{\hbox{\check@mathfonts 
      \fontsize\sf@size\z@ 
      \math@fontsfalse\selectfont 
      A}% 
    \vss}% 
  }% 
  \kern\xxt@kern@aT
  \TeX}}  
\DeclareRobustCommand\XeTeX{%
  \leavevmode
  \smash{%
   X\lower\xxt@lower@e
   \hbox{\kern\xxt@kern@eX
     \ifnum\XeTeXfonttype\font>0
       \ifnum\XeTeXcharglyph"018E>0
         \char"018E\relax
       \else
         \ifdim\fontdimen1\font=0pt
           \reflectbox{E}%
         \else
           \XeTeXuseglyphmetrics=1%
           \setbox0=\hbox{E}\dimen0=\ht0\advance\dimen0by\dp0%
           \raise\dimen0\hbox{\rotatebox{180}{\box0}}%
         \fi
       \fi
     \else
       \setbox0=\hbox{E}\dimen0=\ht0\advance\dimen0by\dp0%
       \raise\dimen0\hbox{\rotatebox{180}{\box0}}%
     \fi
   }\kern\xxt@kern@Te\TeX}}%
\DeclareRobustCommand\XeLaTeX{%
   \leavevmode
   \smash{%
    X\lower\xxt@lower@e
    \hbox{\kern\xxt@kern@eX
      \ifnum\XeTeXfonttype\font>0\relax
        \ifnum\XeTeXcharglyph"018E>0\relax
          \char"018E\relax
        \else
          \ifdim\fontdimen1\font=0pt\relax
            \reflectbox{E}%
          \else
            \XeTeXuseglyphmetrics=1\relax
            \setbox0=\hbox{E}\dimen0=\ht0\advance\dimen0by\dp0\relax
            \raise\dimen0\hbox{\rotatebox{180}{\box0}}%
          \fi
        \fi
      \else
        \setbox0=\hbox{E}\dimen0=\ht0\advance\dimen0by\dp0\relax
        \raise\dimen0\hbox{\rotatebox{180}{\box0}}%        
      \fi}\kern\xxt@kern@eL\LaTeX}}
\TeX@logo@spacing{-0.15em}{-0.15em}{0.5ex}{-0.36em}{-0.15em}{-0.1em}
%    \end{macrocode}
% \changes{v0.1}{2006/05/03}{Implemented.}
% \changes{v0.2}{2006/06/11}{Added \cmd\XeLaTeX\ logo. Thanks Herb!}
% \changes{v0.2}{2006/06/11}{Added TFM font check.}
% \changes{v0.2}{2006/08/12}{\cmd\TeX@logo@spacing made ``private'' and added an arg for \cmd\XeLaTeX.}
% \end{macro}
%
% \section{\eTeX\ functionality}
% Because it's just sensible, we load the package that actually
% allows \LaTeX\ to access the extra registers, etc., provided
% by \eTeX.
%    \begin{macrocode}
\RequirePackage{etex}
%    \end{macrocode}
%
% \section{Unicode footnote symbols}
% \LaTeX\ defines footnote symbols with LICRs that don't resolve well 
% with the \pkg{xunicode} package; better results can be achieved by
% using specific unicode characters.
%
% This problem is solved by the \pkg{fixltx2e} package.
%    \begin{macrocode}
\RequirePackage{fixltx2e}[2006/03/24]
%    \end{macrocode}
%
% \section{\cmd\eminnershape}
% \begin{macro}{\em}
% \begin{macro}{\emph}
% \pkg{fixltx2e}'s method for checking for ``inner'' emphasis is a little fragile in \XeTeX, because
% font slant information might be missing from the font. Therefore, we use \LaTeX's \acro{NFSS}
% information, which is more likely to be correct.
%
% \begin{example}
%   \renewcommand\eminnershape{\scshape}
%   \fontspec{Didot}
%   Nested {\em emphasis is \emph{now} fixed.}
% \end{example}
%
%    \begin{macrocode}
\DeclareRobustCommand\em 
  {\@nomath\em 
   \edef\@tempa{\f@shape}%
   \edef\@tempb{\itdefault}%
   \ifx\@tempa\@tempb 
     \eminnershape 
   \else 
     \emshape 
   \fi}
\DeclareTextFontCommand{\emph}{\em}
\let\emshape\itshape
\let\eminnershape\upshape
%    \end{macrocode}
% \changes{v0.2}{2006/08/12}{Migrated from \textsf{fontspec}.}
% \end{macro} \end{macro}
%
%
% \section{Programmming bits and pieces}
%
% Thanks to a long-ago c.t.t.\ post by Robin Fairbairns for the code how to \cmd\let\ a robust macro.
%    \begin{macrocode}
\newcommand*\robust@let@nc[2]{%
  \expandafter\let\expandafter#1\csname #2 \endcsname}
%    \end{macrocode}
%
% \section{\cmd\-}
% \LaTeX\ defines the macro \cmd\-\ to insert discretionary hyphenation points.
% However, it is hard-coded to use the hyphen |-| character. Since \pkg{fontspec}
% makes it easy to change the hyphenation character on a per font basis, it would
% be nice if \cmd\-\ adjusted automatically.
%
% \begin{macro}{\-}
% This macro is courtesy of Frank Mittelbach and the \LaTeXe\ source code.
%    \begin{macrocode}
%\CheckCommand\-{\discretionary{-}{}{}}
\DeclareRobustCommand{\-}{%
  \discretionary{%
    \char\ifnum\hyphenchar\font<\z@
           \xlx@defaulthyphenchar
         \else
           \hyphenchar\font
         \fi}{}{}}
\def\xlx@defaulthyphenchar{`\-}
%    \end{macrocode}
% \changes{v0.1}{2006/05/03}{Implemented; from the \LaTeXe\ sources.}
% \end{macro}
%
% \section{\cmd\textsuperscript\ and \cmd\textsubscript}
%
% The new macros now allow real text inferiors and superiors:
% \begin{quotation}
%	\fontspec{Hoefler Text}
%	  |\textsuperscript     | \textsuperscript{abcdefghijklmnopqrstuvwxyz1234567890}\par
%	  |\textsubscript       | \textsubscript{abcdefghijklmnopqrstuvwxyz1234567890}
% \end{quotation}
% As opposed to fake ones:
% \begin{quotation}
%	\fontspec{Hoefler Text}
%	  |\textsuperscript*    | \textsuperscript*{abcdefghijklmnopqrstuvwxyz1234567890}\par
%	  |\textsubscript*      | \textsubscript*{abcdefghijklmnopqrstuvwxyz1234567890}
% \end{quotation}
% Or:
% \begin{quotation}
%	\fontspec{Hoefler Text}
%	  |\faketextsuperscript | \textsuperscript*{abcdefghijklmnopqrstuvwxyz1234567890}\par
%	  |\faketextsubscript   | \textsubscript*{abcdefghijklmnopqrstuvwxyz1234567890}
% \end{quotation}
% But beware fonts lacking the full repertoire: (this is Adobe Jenson Pro)
% \begin{quotation}
%	\fontspec{Adobe Jenson Pro}
%	  |\textsuperscript     | \textsuperscript{abcdefghijklmnopqrstuvwxyz1234567890}\par
%	  |\textsubscript       | \textsubscript{abcdefghijklmnopqrstuvwxyz1234567890}
% \end{quotation}
%
% For OpenType fonts, the subscript feature (|subs|) is used, but if that doesn't
% exist then the scientific inferior feature (|sinf|) is used on the assumption
% that something's better than nothing. This assumption may prove to be a poor one,
% and the functionality of the package may change in the future.
%
% \begin{macro}{\faketextsubscript}
% \begin{macro}{\faketextsuperscript}
% The old (`fake') methods:
%    \begin{macrocode}
\robust@let@nc\faketextsubscript{textsubscript} 
\robust@let@nc\faketextsuperscript{textsuperscript} 
%    \end{macrocode}
% \changes{v0.1}{2006/05/03}{Implemented.}
% \end{macro}
% \changes{v0.1}{2006/05/03}{Implemented.}
% \end{macro}
% \begin{macro}{\textsubscript}
% \begin{macro}{\textsubscript*}
% Text subscripts:
%    \begin{macrocode}
\DeclareRobustCommand*\textsubscript{%
  \@ifstar{\faketextsubscript}{\@@textsubscript}}
\newcommand\@@textsubscript[1]{%
  \begingroup
    \c@zf@script 1818326126\relax
    \font\zf@basefont="\csname zf@family@fontdef\f@family\endcsname" at \f@size pt
    \zf@set@font@type
    \ifzf@atsui
      \zf@make@aat@feature@string{10}{2}%
      \unless\ifx\zf@thisfontfeature\@empty
        {\addfontfeature{VerticalPosition=Inferior}#1}%
      \else
        \faketextsubscript{#1}%
      \fi
    \fi
    \ifzf@icu
      \zf@check@ot@feat{+subs}%
      \if@tempswa
        {\addfontfeature{VerticalPosition=Inferior}#1}%
      \else
        \zf@check@ot@feat{+sinf}%
        \if@tempswa
          {\addfontfeature{VerticalPosition=ScientificInferior}#1}%
        \else
          \faketextsubscript{#1}%
        \fi
      \fi
    \fi
  \endgroup}
%    \end{macrocode}
% \changes{v0.1}{2006/05/03}{Implemented.}
% \end{macro}
% \changes{v0.1}{2006/05/03}{Implemented.}
% \end{macro}
% \begin{macro}{\textsuperscript}
% \begin{macro}{\textsuperscript*}
% Text superscripts:
%    \begin{macrocode}
\DeclareRobustCommand*\textsuperscript{%
  \@ifstar{\faketextsuperscript}{\@@textsuperscript}}
\newcommand\@@textsuperscript[1]{%
  \begingroup
    \c@zf@script 1818326126\relax
    \font\zf@basefont="\csname zf@family@fontdef\f@family\endcsname" at \f@size pt
    \zf@set@font@type
    \ifzf@atsui
      \zf@make@aat@feature@string{10}{1}%
      \unless\ifx\zf@thisfontfeature\@empty
        {\addfontfeature{VerticalPosition=Superior}#1}%
      \else
        \faketextsuperscript{#1}%
      \fi
    \fi
    \ifzf@icu
      \zf@check@ot@feat{+sups}%
      \if@tempswa
        {\addfontfeature{VerticalPosition=Superior}#1}%
      \else
        \faketextsuperscript{#1}%
      \fi
    \fi
  \endgroup}
%    \end{macrocode}
% \changes{v0.1}{2006/05/03}{Implemented.}
% \end{macro}
% \changes{v0.1}{2006/05/03}{Implemented.}
% \end{macro}
%
% \section{\cmd\vfrac}
% A command for setting vulgar fractions based on AAT or OpenType font features. 
% Not really recommended for many purposes, depending on your text, but it's a good example.
% \begin{example}
%   \fontspec{Hoefler Text}
%   AAT: \vfrac{123}{456}\par
%   \fontspec{Warnock Pro}
%   ICU: \vfrac{123}{456}
% \end{example}
% \begin{macro}{\vfrac}
% \darg{Numerator}
% \darg{Denominator}
% No error checking is done to ensure that the font actually has the neceessary features.
% Requires the \pkg{xunicode} package for \cmd\textfractionsolidus.
%    \begin{macrocode}
\newcommand*\vfrac[2]{%
  \begingroup
    \c@zf@script 1818326126\relax
    \font\zf@basefont="\csname zf@family@fontdef\f@family\endcsname" at \f@size pt
    \zf@set@font@type
    \ifzf@atsui
      {\addfontfeature{VerticalPosition=Superior}#1}%
       \textfractionsolidus
      {\addfontfeature{VerticalPosition=Inferior}#2}%
    \fi
    \ifzf@icu
      {\addfontfeature{VerticalPosition=Numerator}#1}%
       \textfractionsolidus
      {\addfontfeature{VerticalPosition=Denominator}#2}%      
    \fi
  \endgroup}
%    \end{macrocode}
% \changes{v0.1}{2006/05/03}{Implemented.}
% \end{macro}
%
% \section{Named glyphs}
% Along the way somewhere, \XeTeX\ added support for selecting glyphs from a TrueType-based OpenType font based on their internal glyph name. Jonathan Kew posted the following definition as a nice interface to it.
% \begin{example}
%   \fontspec{Charis SIL}
%   \namedglyph{yen}
%   \namedglyph{smile}
% \end{example}
% \begin{macro}{\namedglyph}
% \darg{Name of the font glyph to be typeset}
%    \begin{macrocode}
\newcommand\namedglyph[1]{%
  \@tempcnta=\XeTeXglyphindex "#1"\relax
  \ifnum\@tempcnta>0 
    \XeTeXglyph\@tempcnta 
  \else 
    \xxt@namedglyph@fallback{#1}%
  \fi}
%    \end{macrocode}
% \changes{v0.2}{2006/08/12}{Implemented.}
% \end{macro}
% \begin{macro}{\xxt@namedglyph@fallback}
% Redefine this macro to change how glyph names that aren't found get typeset.
%    \begin{macrocode}
\newcommand\xxt@namedglyph@fallback[1]{[#1]}
%    \end{macrocode}
% \changes{v0.2}{2006/08/12}{Implemented.}
% \end{macro}
%
%\iffalse
%</package>
%\fi
%
% \clearpage
% \PrintChanges
%
% \clearpage
% \PrintIndex
%
% \Finale
%
%\iffalse
%<*dtx-style>
%    \begin{macrocode}
\ProvidesPackage{dtx-style}

\def\@dotsep{1000}
\setcounter{tocdepth}{2}
\setlength\columnseprule{0.4pt}
\renewcommand\tableofcontents{\relax
  \begin{multicols}{2}[\section*{\contentsname}]\relax
    \@starttoc{toc}\relax
  \end{multicols}}

\setcounter{IndexColumns}{2}
\renewenvironment{theglossary}
  {\small\list{}{}
     \item\relax
     \glossary@prologue\GlossaryParms 
     \let\item\@idxitem \ignorespaces 
     \def\pfill{\hspace*{\fill}}}
  {\endlist}

\usepackage{array,booktabs,calc,color,enumitem,fancyvrb,graphicx,ifthen,longtable,refstyle,varioref}
\usepackage{fontspec,ifxetex,xunicode,xltxtra}
\setromanfont[WordSpace=0.9,Mapping=tex-text,
  BoldFont=Hoefler Text,
  ItalicFeatures={Contextuals={NoWordFinal,NoLineFinal}}]{Hoefler Text}
\setsansfont[Scale=MatchLowercase,Mapping=tex-text,BoldFont=Gill Sans]{Gill Sans}
\setmonofont[Scale=MatchLowercase]{Andale Mono}

\linespread{1.05}      % A bit more space between lines
\frenchspacing         % Remove ugly extra space after punctuation
  
\definecolor{niceblue}{rgb}{0.2,0.4,0.6}
\newenvironment{example}
  {\VerbatimEnvironment
   \par\noindent\hrulefill\nobreak\par\nobreak
   \begin{VerbatimOut}[gobble=4]{\examplefilename}}
  {\end{VerbatimOut}\relax
   \begingroup
     \color{niceblue}
     \typesetexample
   \endgroup\nobreak\par\nobreak\noindent\hrulefill\nobreak\par}

\let\examplesize\normalsize
\let\auxwidth\relax

\newlength\examplewidth\newlength\verbatimwidth
\newlength\exoutdent   \newlength\exverbgap
\setlength\exverbgap{0em}
\setlength\exoutdent{-0\textwidth}
\newsavebox\verbatimbox
\edef\examplefilename{\jobname.example}

\newcommand\typesetexample{\relax
   \begin{lrbox}{\verbatimbox}\relax
     \BVerbatimInput[fontsize=\small]{\examplefilename}\relax
   \end{lrbox}
   \begin{list}{}{\setlength\itemindent{0pt}
                  \setlength\leftmargin\exoutdent
                  \setlength\rightmargin{0pt}}\item
   \ifx\auxwidth\relax
     \setlength\verbatimwidth{\wd\verbatimbox}\relax
   \else
     \setlength\verbatimwidth{\auxwidth}\relax
   \fi
   \begin{minipage}[c]{\textwidth-\exoutdent-\verbatimwidth-\exverbgap}
     \catcode`\%=14\centering\examplesize\input\examplefilename\relax
   \end{minipage}\hfill
   \begin{minipage}[c]{\verbatimwidth}
     \usebox\verbatimbox
   \end{minipage}
   \end{list}
   \global\let\examplesize\normalsize}

\newcommand*\setexsize[1]{\let\examplesize#1}
\newcommand*\setverbwidth[1]{\def\auxwidth{#1}}

\newcommand*\name[1]{{#1}}
\newcommand*\pkg[1]{\textsf{#1}}

\newcommand*\acro[1]{\textsc{\MakeLowercase{#1}}}

\newcommand*\note[1]{\unskip\footnote{#1}}

\let\latin\textit
\def\eg{\latin{e.g.}}
\def\Eg{\latin{E.g.}}
\def\ie{\latin{i.e.}}
\def\etc{\@ifnextchar.{\latin{etc}}{\latin{etc.}\@}}

\def\OMEGA{Ω}

\newcounter{argument}
\g@addto@macro\endmacro{\setcounter{argument}{0}}
\newcommand*\darg[1]{%
  \stepcounter{argument}%
  \noindent{\ttfamily\#\theargument}:~#1\par\nobreak}
\newcommand*\doarg[1]{%
  \stepcounter{argument}%
  \noindent{\ttfamily\makebox[0pt][r]{[}\#\theargument]}:~#1\par\nobreak}
%    \end{macrocode}
%</dtx-style>
%\fi
%
% \typeout{*************************************************************}
% \typeout{*}
% \typeout{* To finish the installation you have to move the following}
% \typeout{* file into a directory searched by XeTeX:}
% \typeout{*}
% \typeout{* \space\space\space xltxtra.sty}
% \typeout{*}
% \typeout{*************************************************************}
%
\endinput
 